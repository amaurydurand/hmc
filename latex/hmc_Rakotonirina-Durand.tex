\documentclass[10pt,a4paper]{article}
\usepackage{init}
\usepackage{defs}
\usepackage{color}
\newcommand{\red}{\color{red}}
\definecolor{gray}{rgb}{0.4, 0.4, 0.4}

\title{Introduction aux méthodes de Monte Carlo par dynamique Hamiltonienne}
\author{Shmuel RAKOTONIRINA-RICQUEBOURG, Amaury DURAND}
\begin{document}
\maketitle
\tableofcontents
\section{Introduction}
Ce rapport présente le travail effectué lors d'un projet du cours d'approfondissements en chaînes de Markov par Eric Moulines dans le cadre du master Mathématiques de l'aléatoires à l'université Paris-Sud. Le contenu présenté ci-dessous repose essentiellement sur \cite{Neal-hmc} et \cite{Douc-mc}. 

Les méthodes de Monte Carlo par dynamique Hamiltonienne, plus communément appelées Hamiltonian Monte Carlo (HMC), font partie d'une grande famille de méthodes de simulation : les méthodes de Monte Carlo, et plus précisemment dans la famille des méthodes de Monte Carlo par chaînes de Markov (ou Monte Carlo Markov Chains). Ces méthodes se placent dans le cadre suivant :
\begin{Def}[Carde général]
	Soit $(\xset, \calX)$ un espace mesurable. Soit $\pi$ une loi de probabilité sur cet espace. On suppose que $\pi$ n'est connue qu'à un facteur de proportionnalité près i.e on connait $\lambda \pi$ où $\lambda \in \rset$ une constante. Le but est de pouvoir approcher $\pi f = \EE[f(X)]$ avec $X \sim \pi$ et $f \in \fset_+(\xset, \calX) \cup \fset_b(\xset, \calX)$.
\end{Def}

\subsection{Méthodes de Monte Carlo}

La méthode de Monte Carlo la plus simple (appelée Monte Carlo naïf) est la suivante : supposons que l'on sait simuler des variables aléatoires de loi $\pi$ alors en tirant $n$ échantillons $X_1, \cdots, X_n \siid \pi$, la loi des grands nombre nous indique qu'une bonne approximation de $\pi f$ est $\frac{1}{n} \sum_{i=1}^n f(X_i)$. D'autres méthodes permettent d'atteindre le même but en ne sachant pas simuler de variables aléatoires de loi $\pi$. C'est le cas par exemple de l'échantillonnage d'importance qui consiste à simuler des variables i.i.d sous une loi différente de $\pi$.

Dans ces deux cas, l'approximation de $\pi f$ repose sur une simulation de variables aléatoires i.i.d. Cette particularité permet alors de montrer des résultats de convergence notamment grâce à la loi des grands nombre. Les méthodes de Monte Carlo par chaînes de Markov ne reposent pas sur le caractère i.i.d des variables mais sur les propriétés des Chaînes de Markov. 

\subsection{Monte Carlo Markov Chains (MCMC)}

Les méthode MCMC se basent sur les notions de loi invariante, de réversibilité et d'ergodicité.

\subsubsection{Loi invariante et réversibilité}

On considère $P$ un noyau de Markov sur $\xset \times \calX$.
\begin{Def}
	Une loi de probabilité $\pi$ sur $(\xset, \calX)$ est dite
	\begin{itemize}
		\item $P$-invariante si $\pi P = \pi$
		\item $P$-réversible si $\forall A,B \in \calX, \pi \otimes P (A \times B) = \pi \otimes P (B \times A)$
	\end{itemize}
\end{Def}

\begin{Prop}
	Toute probabilité $P$-réversible est $P$-invariante.
	% Soit $\pi$ une loi de probabilité sur $(\xset, \calX)$ alors 
	% $$
	% \pi \text{ est } P\text{-reversible} \Rightarrow \pi \text{ est } P\text{-invariante}
	% $$
\end{Prop}

\subsubsection{Ergodicité}

% \begin{Def}[Système ergodique]
%   Soit $(\Omega, \calB, \PP)$ un espace de probabilité. Soit $T : (\Omega, \calB) \to (\Omega, \calB)$ une application mesurable. 
%   \begin{itemize}
%   \item On dit que $\PP$ est invariante pour $T$ si $\forall A \in \calB, \PP[T^{-1}(A)] = \PP[A]$. Dans ce cas, on dit que $(\Omega, \calB, \PP, T)$ est un système dynamique.
%   \item $A \in \calB$ est dit invariant pour $T$ si $A = T^{-1}(A)$.
%   \item Si pour tout $A$ invariant pour $T$ on a $\PP[A] \in \ens{0,1}$ alors on dit que $(\Omega, \calB, \PP, T)$ est un système ergodique. 
%   \end{itemize}
% \end{Def}

% \begin{Thm}\label{thm:ergodic}
%   Soit $P$ un noyau de Markov sur $\xset \times \calX$ et on se place sur l'espace canonique $(\xset^n, \calX^{\otimes n})$. On note $ \theta : \fundef{ \xset^\nset &\to& \xset^\nset \\ (\omega_t)_{t\in \nset} &\mapsto& (\omega_{t+1})_{t \in \nset}}$ et  $\forall k \in \nset^*, \theta_k = \theta_{k-1} \circ \theta$ avec $\theta_0 = \Id$. On suppose
%   \begin{enumerate}
%   \item $P$ possède une loi invariante $\pi$
%   \item $(\xset^\nset, \calX^{\otimes \nset}, \PP_\pi, \theta)$ est ergodique
%   \end{enumerate}
%   Alors pour toute v.a $Y \in L^1(\xset^\nset, \calX^{\otimes \nset}, \PP_\pi)$, pour $\pi$-presque tout $x \in \xset$, 
% $$
%   \frac{1}{n} \sum_{k=0}^{n-1} Y \circ \theta_k \xrightarrow[n \to +\infty]{\PP_x\text{-}\ps} \EE{\pi}[Y]
%   $$
% \end{Thm}

% \begin{Rque}
%   Considérons $(X_k)_{k \in \nset}$ la chaîne de Markov canonique de noyau $P$ et $f \in \fset_+(\xset, \calX) \cup \fset_b(\xset, \calX)$ alors en prenant $Y = f(X_0)$, on a $Y \circ \theta_k = f(X_k)$ et $\EE{\pi}[Y] = \EE{\pi}[f(X_0)] = \pi f$ donc le résultat du théorème \ref{thm:ergodic} se réécrit : pour $\pi$-presque tout $x \in \xset$, 
% $$
%   \frac{1}{n} \sum_{k=0}^{n-1} f(X_k) \xrightarrow[n \to +\infty]{\PP_x\text{-}\ps} \pi f
%   $$
%   Ce qui montre que l'on peut avoir une bonne approximation de $\pi f$ en construisant une chaîne de Markov. De plus il est intéressant de constater que la convergence est $\PP_x$ presque sûre pour $\pi$-presque tout $x \in \xset$ ce qui signifie que quelque soit le point de départ, on est sûr d'avoir une bonne approximation de $\pi f$ si on attend suffisamment longtemps.
% \end{Rque}

\begin{Thm}\label{thm:ergodic}
	Soit $(X_k)_{k \in \nset}$ une chaîne de Markov de noyau $P$ admettant une unique loi invariante $\pi$. Alors pour tout $f \in \fset_+(\xset, \calX) \cup \fset_b(\xset, \calX)$ et pour $\pi$-presque tout $x \in \xset$,
	$$\frac{1}{n} \sum_{k=0}^{n-1} f(X_k) \xrightarrow[n \to +\infty]{\PP_x\text{-}\ps} \pi f$$
\end{Thm}

Les algorithmes MCMC visent à construire, à partir de $\pi$, une chaîne de Markov vérifiant les hypothèses de ce théorème afin d'approcher $\pi f$ par $\frac{1}{n} \sum_{k=0}^{n-1} f(X_k)$.

\subsubsection{Algorithme de Metropolis (Random Walk Metropolis)}

Avant de présenter la méthode HMC, nous définissons ici un autre algorithme MCMC, l'algorithme de Metropolis, que nous utiliserons comme base de comparaison. On se place dans le cas où $\xset = \rset^d$ et $\calX = \calB(\rset^d)$. 

On suppose que $\pi$ a une densité $h_\pi$ par rapport à une mesure $\mu$. On considère de plus un loi $Q$ (appelée loi instrumentale) sur $(\xset, \calX)$ symétrique par rapport à $0$ ($Q(A) = Q(-A)$. La construction de la chaîne de Markov se fait en proposant un mouvement (dont l'incrément suit $Q$), puis en acceptant ou en rejetant ce mouvement (algorithme \ref{algo:metropolis}).

\begin{center}
	\begin{algorithm}[H]
		\KwData{$h_\pi$ proportionnel à la densité cible, $Q$ loi simulable}
		$X_0 \leftarrow x \in \xset$ arbitraire\;
		$(U_k)_{k \in \nset} \siid Q$ \;
		\Repeat{une condition d'arrêt}{
			$Y_{k+1} \leftarrow X_k + U_{k+1}$ \tcp*{Proposer un mouvement}
			$\alpha_{k+1} \leftarrow \alpha(X_k, Y_{k+1})$ où $\alpha(x,y) = 1 \wedge \frac{h_{\pi}(y)}{h_\pi(x)}$\;
			$X_{k+1} \leftarrow \piecewise{ Y_{k+1} & \text{avec probabilité } \alpha_{k+1} \\ X_k & \text{with probability } 1 - \alpha_{k+1}}$ \tcp*{Accepter ou rejeter le mouvement}
		}
		\KwRet{$(X_k)_k$}
		\caption{Random Walk Metropolis}
		\label{algo:metropolis}
	\end{algorithm}
\end{center}

Dans cet algorithme, on cherche à explorer l'espace $\xset$ en visitant moins souvent les régions où $h_\pi$ est faible (peu chargées par $\pi$) car ce sont les régions de forte probabilité qui donnent des informations sur la loi $\pi$. Ainsi, si le mouvement proposé vérifie $h_\pi(Y_{k+1}) \geq h_\pi(X_k)$, on se dirige vers une zone plus chargée par $\pi$, on accepte donc le mouvement. Dans le cas contraire, il est moins intéressant de bouger. On autorise quand même le mouvement avec une probabilité $\frac{h_\pi(Y_{k+1})}{h_\pi(X_k)}$ d'autant plus faible que la position proposée est dans une région de probabilité faible.

% \subsubsection{Algorithme de Metropolis-Hastings}

% L'algorithme HMC est une version améliorée de l'algorithme de Metropolis-Hastings. Nous allons donc d'abord définir ce dernier et nous l'utiliserons plus tard comme base de comparaison. On se place dans le cas où $\xset = \rset^d$ et $\calX = \calB(\rset^d)$. 

% On suppose que $\pi$ a une densité $h_\pi$ par rapport à une mesure $\mu$. On considère de plus un noyau markovien $Q$ sur $(\xset, \calX)$ de densité $q$ par rapport à $\mu$. $Q$ est appelé noyau instrumental. La construction de la chaîne de Markov se fait en proposant un mouvement via $Q$, puis en acceptant ou en rejetant ce mouvement (algorithme \ref{algo:metropolis-hastings}).

% \begin{center}
% 	\begin{algorithm}[H]
% 		\KwData{$h_\pi$ proportionnel à la densité cible, $Q$ noyau markovien simulable}
% 		$X_0 \leftarrow x \in \xset$ arbitraire\;
% 		\Repeat{une condition d'arrêt}{
% 			$Y_{k+1} \sim Q(X_k,\cdot)$ \tcp*{Proposer un mouvement}
% 			$\alpha_{k+1} \leftarrow \alpha(X_k, Y_{k+1})$ où $\alpha(x,y) = 1 \wedge \frac{h_{\pi}(y)q(y,x)}{h_\pi(x)q(x,y)}$\;
% 			$X_{k+1} \leftarrow \piecewise{ Y_{k+1} & \text{avec probabilité } \alpha_{k+1} \\ X_k & \text{avec probabilité } 1 - \alpha_{k+1}}$ \tcp*{Accepter ou rejeter le mouvement}
% 		}
% 		\KwRet{$(X_k)_k$}
% 		\caption{Metropolis-Hastings}
% 		\label{algo:metropolis-hastings}
% 	\end{algorithm}
% \end{center}

% Dans cet algorithme, on cherche à explorer l'espace $\xset$ en visitant moins souvent les régions où $h_\pi$ est faible (peu chargées par $\pi$) car ce sont les régions de forte probabilité qui donnent des informations sur la loi $\pi$. Ainsi, si le mouvement proposé vérifie $h_\pi(Y_{k+1})q(Y_{k+1}, X_k) \geq h_\pi(X_k) q(X_k, Y_{k+1})$, on se dirige vers une zone plus chargée par $\pi$, on accepte donc le mouvement. Dans le cas contraire, il est moins intéressant de bouger. On autorise quand même le mouvement avec une probabilité $\frac{h_\pi(Y_{k+1})q(Y_{k+1},X_k)}{h_\pi(X_k)q(X_k,Y_{k+1})}$ d'autant plus faible que la position proposée est dans une région de probabilité faible.

% \begin{Rque}
% 	Il y a d'autres fonctions de rejet $\alpha$ qui permettent à l'algorithme de fonctionner. De même, le choix du noyau $Q$ est un degré de liberté de l'algorithme.

% 	Le choix de $Q$ est en fait une difficulté de l'algorithme, puisque le mouvement proposé doit permettre l'exploration de tout l'espace (afin de passer régulièrement par toutes les régions fortement chargées par $\pi$). Nous verrons que l'algorithme HMC résout ce problème en proposant le mouvement selon une \emph{dynamique hamiltonienne}.
% \end{Rque}

% \begin{Def}
% 	On parle de \emph{Random-walk Metropolis} quand $Q$ est le noyau d'une marche aléatoire
% 	$$Y_{k+1} \leftarrow X_k + U_{k+1}$$
% 	avec $(U_k)_{k \in \nset}$ iid. On supposera que $Q$ admet une densité $q$ par rapport à $\mu$ avec $q(x) = q(-x)$.
% \end{Def}

% \begin{center}
% 	\begin{algorithm}[H]
% 		\KwData{$h_\pi$ proportionnel à la densité cible, $Q$ loi simulable}
% 		$X_0 \leftarrow x \in \xset$ arbitraire\;
% 		$(U_k)_{k \in \nset} \siid Q$ \;
% 		\Repeat{une condition d'arrêt}{
% 			$Y_{k+1} \sim X_k + U_{k+1}$ \tcp*{Proposer un mouvement}
% 			$\alpha_{k+1} \leftarrow \alpha(X_k, Y_{k+1})$ où $\alpha(x,y) = 1 \wedge \frac{h_{\pi}(y)}{h_\pi(x)}$\;
% 			$X_{k+1} \leftarrow \piecewise{ Y_{k+1} & \text{avec probabilité } \alpha_{k+1} \\ X_k & \text{avec probabilité } 1 - \alpha_{k+1}}$ \tcp*{Accepter ou rejeter le mouvement}
% 		}
% 		\KwRet{$(X_k)_k$}
% 		\caption{Random Walk Metropolis}
% 		\label{algo:metropolis}
% 	\end{algorithm}
% \end{center}

\subsection{Principe du Hamiltonian Monte Carlo}

On se place dans le cas où $\xset = \rset^d$, $\calX = \calB(\rset^d)$ et $\mu$ est la mesure de Lebesgue. 

L'algorithme HMC fait partie des algorithmes MCMC, le but est en effet de construire une chaîne de Markov de loi invariante la loi à simuler. Il diffère néanmoins de l'algorithme de Metropolis-Hastings notamment par le fait que le mouvement proposé suit une \emph{dynamique hamiltonienne} associée à la loi à simuler.

\subsubsection{Dynamique Hamiltonienne}

Un système physique de position $x : \rset_+ \to \rset^d$ et de quantité de mouvement $p : \rset_+ \to \rset^d$ (toutes deux fonctions du temps $t$) est caractérisé par des équations de mouvement de type $\forall t \geq 0, \forall i \in \iseg{1,d}$, 
\begin{equation}\label{eq:hamiltonian-dyn}
	\begin{aligned}
		x_i'(t) &= \frac{\partial H}{\partial p_i} (x(t), p(t)) \\
		p_i' (t) &= -\frac{\partial H}{\partial x_i} (x(t), p(t))
	\end{aligned}
\end{equation}
où $H : \fundef{\rset^d \times \rset^d & \to & \rset \\ (x,p) & \mapsto & H(x,p)}$ s'appelle le {\it hamiltonien} du système.

Le Hamiltonien s'interprète comme une énergie sur $x$ et $p$. En général $H(x,p) = U(x) + K(p)$ où $U$ est l'énergie potentielle et $K$ l'énergie cinétique.

\begin{Rque}\label{rque:edo}
	Les équations \eqref{eq:hamiltonian-dyn} se réécrivent comme une équation différentielle ordinaire d'ordre 1 :
	\begin{equation}\label{eq:edo}
		\forall t \geq 0, \quad z'(t) = F(z(t))
		\tag{EDO}
	\end{equation}
	avec $z : \fundef{ \rset_+ & \to & \rset^{2d} \\ t & \mapsto & [x_1(t), \cdots, x_d(t), p_1(t), \cdots, p_d(t)]^\top}$ et $\forall z \in \rset^{2d}, F(z) = J \nabla H (z)$ où
	$J = \begin{bmatrix}
		0_{d} & I_{d} \\
		-I_{d} & 0_{d}
	\end{bmatrix}$.
	Une solution de \eqref{eq:edo} est dite solution du hamiltonien $H$.
\end{Rque}

\subsubsection{Principe général}

La dynamique hamiltonienne décrit le mouvement d'un objet qui glisse sans frottement le long d'une surface ou d'une courbe. L'objet est décrit par sa position $x \in \rset^d$ et sa quantité de mouvement $p \in \rset^d$ et on lui associe des énergies potentielle $U(x)$ et cinétique $K(p)$.

\begin{Def}[Distribution canonique]
	En physique, une énergie $E : \rset^n \to \rset$ et une température $T > 0$ sont associées à une loi de probabilité par la loi de Boltzmann (ou distribution canonique) de densité par rapport à la mesure de Lebesgue sur $\rset^n$,
	$$\forall x \in \rset^n,  \quad p(x) \propto \exp \left( -\frac{E(x)}{T} \right)$$
\end{Def}


Dans le cas des HMC, on étend l'espace d'états $\xset = \rset^d$ à $\rset^d \times \rset^d$ et on va construire une chaîne de Markov $(Z_k = (X_k, P_k))_{k \in \nset}$ à valeurs dans ce nouvel espace avec $\forall k \in \nset, X_k \indep P_k$ et de loi invariante $\widetilde{\pi} = \pi \otimes \nu$ où $\pi$ est une distribution canonique associée à l'énergie potentielle et $\nu$ est une distribution canonique associée à l'énérgie cinétique. Ainsi leurs densités $h_\pi$ et $h_\nu$ par rapport à la mesure de Lebesgue sur $\rset^d$ vérifient $\forall x \in \rset^d, \forall p \in \rset^d$,
\begin{align*}
	h_\pi(x) & \propto  \exp \left( -\frac{U(x)}{T} \right) \\
	h_\nu(p) & \propto  \exp \left( -\frac{K(p)}{T} \right)
\end{align*}

Ce qui donne une loi jointe de densité $h_{\widetilde{\pi}}$ telle que $\forall (x,p) \in \rset^d \times \rset^d$, 
\begin{equation} \label{eq:canonical-dist}
	h_{\widetilde{\pi}}(x,p) \propto \exp \left( -\frac{H(x,p)}{T} \right)
\end{equation}
où le Hamiltonien est défini par $H : (x,p) \mapsto U(x) + K(p)$.

\begin{Rque}
	En dimension 1, on décrit le glissement sans frottement d'un objet sur une rampe. Comme l'énergie potentielle est proportionnelle à là hauteur de la rampe, \eqref{eq:canonical-dist} donne que les creux de la rampe représentent les régions de forte probabilité. L'objet aura tendance à glisser vers les creux de la rampe, mais peut remonter une pente si sa quantité de mouvement $p$ (et donc son énergie cinétique $K(p)$) est assez grande.
\end{Rque}

L'algorithme HMC propose ses mouvements selon le principe suivant : 
\begin{enumerate}
	\item Considérer un objet placé en $X_k$ de quantité de mouvement $P_k$. Cela correspond à l'état $Z_k = (X_k, P_k)$ de l'objet à l'instant $k$.
	\item Tirer une nouvelle quantité de mouvement $\tilde{P}_k \sim \nu$ indépendante de la chaîne et considérer le nouvel état $\tilde{Z}_k = (X_k, \tilde{P}_k)$. 
	\item A partir de l'état $\tilde{Z}_k$, simuler la dynamique hamiltonienne (déterministe) de l'objet pendant une durée fixée.
	\item Considérer $Z_{k+1}^{\rm prop}$ l'état de l'objet à la fin de la simulation, et accepter ou rejeter le mouvement selon une fonction de rejet $\alpha$ dépendant de $H$. 
\end{enumerate}

Il est important de noter que le HMC n'est pas une méthode de Métropolis-Hastings car, bien que le principe soit similaire (proposer un mouvement et le choisir selon une fonction de rejet dépendant de la loi cible), le HMC tire une nouvelle quantité de mouvement à chaque étape.

Plusieurs problématiques sont alors soulevées : les questions les plus importantes sont de savoir si la loi $\widetilde{\pi}$ est bien invariante et si l'algorithme converge vers cette loi. Nous ne nous intéresserons ici qu'à la question de l'invariance sans résoudre la question de la convergence. De plus, la forme de loi que l'on considère impose des conditions sur les lois $\pi$ et $\nu$, notamment des conditions de régularité de leurs densités (on doit pouvoir dériver le Hamiltonien). Enfin, comment simuler la dynamique hamiltonienne et quelles sont les différences entre le cadre idéal où le mouvement proposé est exactement celui décrit par les équations \eqref{eq:hamiltonian-dyn} et le cadre pratique où l'on doit discrétiser ces équations ?

\paragraph{Hypothèses:}
Afin de pouvoir modéliser le problème par un HMC, on suppose
\begin{equation}\label{eq:hyp}
	\begin{aligned}
		&\mbox{$h_\pi$ et $h_\nu$ sont strictement positives sur $\rset^d$} \\
		&\mbox{$\exists k \geq 1, \ln(h_\pi)$ et $\ln(h_\nu)$ sont de classe $\calC^k$ sur $\rset^d$}\\
		&\mbox{$h_\nu$ est paire} \\
                & H : (x,p) \mapsto U(x) + K(p) \text{ avec } U(x) = - T \ln(h_\pi(x)), K(p) = -T \ln(h_\nu(p))
	\end{aligned}
	\tag{H}
\end{equation}

\begin{Rque}
	La première hypothèse de \eqref{eq:hyp} est nécessaire pour écrire les densités sous la forme canonique. La deuxième permet d'avoir des garanties de régularité des solutions de \eqref{eq:hamiltonian-dyn}. Les deux dernières permettront d'avoir une propriété de réversibilité (proposition \ref{prop:rev}).
\end{Rque}

Nous allons voir que, sous ces hypothèses, dans le cadre idéal (théorique) où l'on a accès aux solutions de \eqref{eq:edo}, le fait de tirer une quantité de mouvement aléatoire puis de suivre la dynamique hamiltonienne laisse la mesure $\tilde{\pi}$ invariante. Dans le cadre où l'on simule la dynamique en discrétisant les équations \eqref{eq:hamiltonian-dyn}, nous verrons que cela n'est plus le cas mais qu'il est alors possible de rendre $\tilde{\pi}$ invariante en incluant le mouvement proposé dans un cadre Métropolis-Hastings (c'est à dire en acceptant le mouvement selon une fonction de rejet). Pour cela, nous avons besoin de quelques propriétés de la dynamique hamiltonienne énoncées dans la section suivante.

%\subsection{Propriétés de l'algorithme}

% Lister les propriétés et bénéfices de HMC en admettant qu'on a les outils de la subsection précédente

\section{Dynamique hamiltonienne}

\subsection{Propriétés de la dynamique hamiltonienne}
\begin{Prop}[conservation du hamiltonien]\label{prop:conservation}
	Le hamiltonien est conservé sur la trajectoire i.e si $z$ est solution de \eqref{eq:edo} alors $H \circ z$ est constant.
\end{Prop}

\begin{proof}
	Remarquons tout d'abord que la matrice $J$ définie dans la remarque \ref{rque:edo} vérifie $\forall u \in \rset^{2d}, \inner{u, Ju} = 0$. Ainsi, si $z$ est solution de \eqref{eq:edo},
	$$
	\forall t \in \rset_+, (H \circ z)'(t) = \inner{\nabla H (z(t)), z'(t)} = \inner{\nabla H (z(t)), J \nabla H(z(t))} = 0.
	$$
\end{proof}

\begin{Def}[flot hamiltonien]
	On définit le flot du hamiltonien par
	$$
	\forall t \in \rset, \phi_t : \fundef{
		\rset^{2d} & \to & \rset^{2d} \\
		z_0 = (x_0, p_0) &\mapsto & \phi_t(z_0) = (\phi_t^{(1)}(z_0), \phi_t^{(2)}(z_0))
	}
	$$
	où $t \mapsto \phi_t(z_0)$ est l'unique solution de \eqref{eq:edo} avec condition initiale $z(0) = z_0$ (en étendant les solutions à tout $\rset$).
\end{Def}

\begin{Pte}\label{prop:compo}
  $$
  \forall t, s \in \rset, \phi_{t+s} = \phi_t \circ \phi_s.
  $$
\end{Pte}
\begin{proof}
  Soient $s \in \rset$ et $z_0 \in \rset^{2d}$, notons $f : t \mapsto \phi_{t}(z_0)$, $h : t \mapsto t+s$ et $g : t \mapsto \phi_t \circ \phi_s (z_0)$. Alors $f$ et $g$ sont solutions de \eqref{eq:edo} avec conditions initiales $f(0) = z_0$ et $g(0) = \phi_s(z_0)$. De plus $\forall t \in \rset, (f\circ h)'(t) = f'(h(t)) = F(f \circ h(t))$ donc $f \circ h$ est aussi solution de \eqref{eq:edo}. Or $f \circ h (0) = \phi_s(z_0) = g(0)$ donc par unicité de la solution $f \circ h = g$ i.e $\forall t \in \rset, \phi_{t+s}(z_0) = \phi_t \circ \phi_s (z_0)$. 
\end{proof}

\begin{Lem}\label{lem:sol-inverse}
  Soit $z = (x,p)$ une solution de \eqref{eq:edo}. On définit $\bar{z} = (\bar x, \bar p) : \fundef{\rset_{-} & \to & \rset^{2d} \\ t & \mapsto & (x(-t), -p(-t))}$ et $\bar H : \fundef{\rset^{2d} & \to & \rset^{2d} \\ (x,p) & \mapsto & H(x,-p)}$. Alors

  \begin{enumerate}
  \item $\bar{z}$ est solution du hamiltonien $\bar H$ sur $\rset_{-}$ (avec d'autres conditions initiales).
  \item Si $\bar{H} = H$ alors $\forall t \in \rset_+, \phi_t(\bar{z}(-t)) = \bar{z}(0)$
  \end{enumerate}
  
\end{Lem}
\begin{proof}
	%D'une part,
	%$$
	%\forall t \in \rset_-, \bar x'(t) = - x'(-t) = - \frac{\partial H}{\partial p}(x(-t),p(-t)) = - \frac{\partial H}{\partial p}(\bar x(t), - \bar p(t)) = \frac{\partial{\bar H}}{\partial p}(\bar x(t), \bar p(t))
	%$$
	%et d'autre part,
	%$$
	%\forall t \in \rset_-, \bar p'(t) = p'(-t) = - \frac{\partial H}{\partial x}(x(-t),p(-t)) = - \frac{\partial H}{\partial x}(\bar x(t), - \bar p(t)) = - \frac{\partial{\bar H}}{\partial x}(\bar x(t), \bar p(t))
	%$$
	\begin{enumerate}
		\item   Notons $s : z = (x,p) \mapsto (x,-p)$ alors $\forall t \in \rset_-, \bar{z}(t) = s \circ z (-t)$ et $\bar{H} = H \circ s$. De plus $\forall z \in \rset^{2d}, \frac{\partial s}{\partial z}(z) = \begin{bmatrix} I_d & 0_d \\ 0_d & -I_d \end{bmatrix} = S$. Remarquons alors que $S J = - J S$ et $s^{-1} = s$. Ainsi, $\forall y \in \rset^{2d}, \nabla \bar{H} (y) = S \nabla H (s(y))$ et pour $t \in \rset_+$,
		  $$
		  \bar{z}'(t) = - S z'(-t) = - S J \nabla H (z(-t)) = J S \nabla H(z(-t)) =  J \nabla \bar{H} (s^{-1}(z(-t))) = J \nabla \bar{H} (s(z(-t))) = J \nabla \bar{H} (\bar{z}(t)).
		  $$
		% \item Si $\bar{H} = H$, $\bar{z}$ est solution de \eqref{eq:edo} sur $\rset_-$. Notons $f$ l'unique solution de \eqref{eq:edo} sur $\rset$ telle que $f(0) = \bar{z}(-t)$ et $h : u \in \rset \mapsto t + u$. Alors $\forall u \in \rset_+, (f\circ h)'(u) = f(h(u)) = F(f\circ h (u))$ donc $f\circ h$ est solution sur $\rset$. Or $f\circ h (-t) = f(0) = \bar{z}(-t)$ donc par unicité de la solution sur $\rset_-$, $(f \circ h)_{| \rset_-} = \bar{z}$. Cela entraine $f(t) = f \circ h (0) = \bar{z}(0)$. Or par définition de $\phi_t$ et unicité de la solution sur $\rset_+$, on a $f_{|\rset_+} : u \in \rset_+ \mapsto \phi_u(\bar{z}(-t))$ donc pour $u = t$ on obtient $\phi_t(\bar{z}(-t)) = f(t) = \bar{z}(0)$
		\item Si $\bar{H} = H$, $\bar{z}$ est solution de \eqref{eq:edo} sur $\rset_-$ donc $\forall t \in \rset_+, \bar z(-t) = \phi_{-t}(\bar z(0))$ donc $\phi_t(\bar{z}(-t)) = \phi_0(\bar{z}(0)) = \bar{z}(0)$
	\end{enumerate}
\end{proof}

\begin{Rque}\label{rque:Hbar}
	Sous les hypothèses \eqref{eq:hyp}, la parité de $h_\nu$ donne la parité de $K$. Comme $H(x,p) = U(x)+K(p)$, on en déduit $H = \bar H$.
\end{Rque}

\begin{Prop}[réversibilité]\label{prop:rev}
	On se place les hypothèses \eqref{eq:hyp}. Alors pour $t \in \rset_+, \phi_t$ est un $\calC^k$-différomorphisme et
	$$
	\phi_t^{-1} : (x,p) \mapsto (\phi_t^{(1)}(x, -p), - \phi_t^{(2)}(x, -p))
	$$
\end{Prop}
\begin{proof}
	$\phi_t$ est $\calC^k$ comme flot d'une équation différentielle $\calC^k$. Si on admet la formule de l'inverse, on y lit que $\phi_t^{-1}$ est $\calC^k$. Il suffit donc de montrer cette formule.

	Fixons $t \in \rset_+$ et $(x_0,p_0) \in \rset^{2d}$. Notons $z=(x,p)$ la solution sur $\rset_+$ de \eqref{eq:edo} avec $z(0) = (x_0,p_0)$ et notons
	$$
	\bar \phi_t : (x,p) \mapsto (\phi_t^{(1)}(x, -p), - \phi_t^{(2)}(x, -p))
	$$
        \begin{itemize}
		\item D'une part,
		$$
		\bar \phi_t(\phi_t(x_0,p_0)) = \bar \phi_t(z(t)) = (\phi_t^{(1)}(\bar z(-t)), - \phi_t^{(2)}(\bar z(-t))) \overset{\text{(lemme \ref{lem:sol-inverse})}}{=} (x_0, p_0)
		$$
		car $\bar z(0) = (x_0,-p_0)$.
		%Par lemme \ref{lem:sol-inverse} et remarque \ref{rque:Hbar}, $\bar z$ est solution sur $\rset_-$ de \eqref{eq:edo} avec $\bar z(0) = (x_0,-p_0)$ donc
		%$$
		%\phi_t^{(1)}(\bar z(-t)) = x_0, \phi_t^{(2)}(\bar z(-t)) = -p_0
		%$$
		%ce qui donne que $\bar \phi_t(\phi_t(x_0,p_0)) = (x_0,p_0)$.

		\item D'autre part,
		$$
		\bar \phi_t(x_0,-p_0) = (\phi_t^{(1)}(x_0, p_0), - \phi_t^{(2)}(x_0, p_0)) = (x(t),-p(t)) = \bar z(-t)
		$$
		donc
		$$
		\phi_t (\bar \phi_t(x_0,-p_0)) = \phi_t(\bar z(-t)) = \bar{z}(0) = (x_0,-p_0).
		$$
	\end{itemize}
	Ceci étant vrai pour tout $(x_0,p_0)$, on a donc $\bar \phi_t = \phi_t^{-1}$.
\end{proof}


\begin{Prop}[conservation du volume]\label{prop:vol}
	$$
	\forall t \in \rset_+, \forall z \in \rset^{2d}, \det \left( \frac{\partial \phi_t}{\partial z}(z) \right) = 1
	$$
\end{Prop}

\subsection{Discrétisation des équations}

% Donner rapidement les résultats déterministes de l'article sur l'hamiltonnien (2.1 et 2.2 de l'article), peut-être qu'on peut se contenter du cas U = -log(h_\pi) et K = \norm{v}/2 (T = 1, m = 1)
% Expliquer la méthode du saute-mouton pour simuler la dynamique hamiltonienne, en quoi elle est meilleure, en quoi elle peut-être simplifiée (2.3 de l'article), éventuellement en quoi elle "préserve le volume"


%\begin{center}
%        \begin{algorithm}[H]
%        	\KwData{pas $\epsilon$, nombre de pas $L$, état initial $(x_0,p_0)$}
%        	$(x,p) \leftarrow (x_0,p_0)$\;
%        	$p \leftarrow p - \frac{\epsilon}{2} \nabla U(x)$ \tcp*{Demi-pas en $p$}
%        	\For(\tcp*[h]{Saute-mouton}){$k \in \llbracket 1,L-1 \rrbracket$}{
%        		$x \leftarrow x + \epsilon p$\;
%        		$p \leftarrow p - \epsilon \nabla U(x)$\;
%        	}
%        	$x \leftarrow x + \epsilon p$ \tcp*{Dernier pas en $x$}
%        	$p \leftarrow p - \frac{\epsilon}{2} \nabla U(x)$ \tcp*{Demi-pas en $p$}
%        	\KwRet{$(x,p)$}
%        	\caption{Discrétisation de l'évolution par saute-mouton (\emph{leapfrog})}
%        	\label{algo:leapfrog}
%        \end{algorithm}
%\end{center}

\begin{center}
	\begin{algorithm}[H]
		\KwData{pas $\epsilon$, nombre de pas $L$, état initial $(x_0,p_0)$}
		\For(\tcp*[h]{Saute-mouton}){$k \in \llbracket 0, L-1 \rrbracket$}{
			$x_{k+1} \leftarrow x_k + \epsilon \nabla K \left( p_k - \frac{\epsilon}{2} \nabla U(x_k) \right)$\;
			$p_{k+1} \leftarrow p_k - \frac{\epsilon}{2} \nabla U(x_k) - \frac{\epsilon}{2} \nabla U(x_{k+1})$\;
		}
		\KwRet{$(x_L,p_L)$}
		\caption{Discrétisation de l'évolution par saute-mouton (\emph{leapfrog})}
		\label{algo:leapfrog}
	\end{algorithm}
\end{center}

\begin{Def}[flot approché] Soient $\epsilon > 0$ et $L \in \nset^*$, on définit le flot approché du hamiltonien par $L$ itérations de l'algorithme leapfrog par $\hat{\phi}_L = \hat{\phi}^L$ où 
  $$
  \hat{\phi} : \fundef{\rset^{2d} & \to & \rset^{2d} \\
    z = (x, p) & \mapsto & (\hat{\phi}^{(1)}(z), \hat{\phi}^{(2)}(z))
    }
  $$
  avec
  \begin{align*}
    \hat{\phi}^{(1)} : (x,p) & \mapsto x + \epsilon \nabla K \left( p - \frac{\epsilon}{2} \nabla U(x) \right) \\
    \hat{\phi}^{(2)} : (x,p) & \mapsto p - \frac{\epsilon}{2} \nabla U(x) - \frac{\epsilon}{2} \nabla U(\hat{\phi}^{(1)}(x,p))
  \end{align*}
\end{Def}

\begin{Prop}[réversibilité]\label{prop:rev-leapfrog}
  Sous les hypothèses \eqref{eq:hyp}, $\hat{\phi}$ est inversible avec $\hat{\phi}^{-1} : (x,p) \mapsto (\hat{\phi}^{(1)}(x,-p), -\hat{\phi}^{(2)}(x,-p))$, ce qui entraîne $\forall L \in \nset^*, \hat{\phi}_L$ est inversible avec $\hat{\phi}_L^{-1} : (x,p) \mapsto (\hat{\phi}_L^{(1)}(x,-p), -\hat{\phi}_L^{(2)}(x,-p))$. 
\end{Prop}
\begin{proof}
	Notons $\bar{\phi} : (x,p) \mapsto (\hat{\phi}^{(1)}(x,-p), -\hat{\phi}^{(2)}(x,-p))$, ainsi $\bar \phi = s \circ \hat \phi \circ s$ avec $s : (x, p) \mapsto (x,-p)$. Alors
	\begin{itemize}
		\item Soient $x,p \in \rset^d$. Notons $(\tilde x, \tilde p) = \hat \phi(x,-p)$ de sorte que $\bar \phi(x,p) = (\tilde x, - \tilde p)$
		\begin{align*}
			\hat{\phi}^{(1)} (\bar{\phi}(x,p))
			&= \tilde x + \epsilon \nabla K \left( -\tilde p - \frac{\epsilon}{2} \nabla U (\tilde x) \right) \\
			&= x + \epsilon \nabla K \left(-p - \frac{\epsilon}{2} \nabla U(x) \right) + \epsilon \nabla K \left(p + \frac{\epsilon}{2} \nabla U(x) + \frac{\epsilon}{2} \nabla U(\tilde x) - \frac{\epsilon}{2} \nabla U (\tilde x) \right) \\
			&= x + \epsilon \nabla K \left(-p - \frac{\epsilon}{2} \nabla U(x) \right) + \epsilon \nabla K \left(p + \frac{\epsilon}{2} \nabla U(x)\right) \\
			&= x \text{ car sous \eqref{eq:hyp}, } K \text{ est paire donc } \nabla K \text{ est impaire.}
		\end{align*}

		De plus
		\begin{align*}
			\hat{\phi}^{(2)}(\bar{\phi}(x,p))
			&= -\tilde p - \frac{\epsilon}{2} \nabla U(\tilde x) - \frac{\epsilon}{2} \nabla U(\hat{\phi}^{(1)}(\bar{\phi}(x,p))) \\
			&= p + \frac{\epsilon}{2} \nabla U(x) + \frac{\epsilon}{2} \nabla U(\tilde x) - \frac{\epsilon}{2} \nabla U(\tilde x) - \frac{\epsilon}{2} \nabla U(x) \\
			&= p
		\end{align*}
		Donc $\hat{\phi} \circ \bar{\phi} = \Id$
		% \item Soient $x,p \in \rset^d$, 
		%   $$
		%   \bar{\phi}^{(1)} (\hat{\phi}(x,p)) = \hat{\phi}^{(1)} (\bar{\phi}(x,-p)) = x 
		%   $$
		%   et
		%   $$
		%   \bar{\phi}^{(2)} (\hat{\phi}(x,p)) = - \hat{\phi}^{(1)} (\bar{\phi}(x,-p)) = -(-p) = p
		%   $$

		%     Donc $\bar{\phi} \circ \hat{\phi} = \Id$
		%   \end{itemize}
		%   Ainsi $\hat{\phi}^{-1} = \bar{\phi}$
		\item On a donc
		$$\Id = \hat \phi \circ \bar \phi = \hat \phi \circ s \circ \hat \phi \circ s.$$
		En composant à gauche et à droite par $s$, on obtient
		$$s^2 = s \circ \hat \phi \circ s \circ \hat \phi \circ s^2$$
		soit, en utilisant $s^2 = \Id$,
		$$\Id = s \circ \hat \phi \circ s \circ \hat \phi.$$
	\end{itemize}
	Ainsi $\hat{\phi}^{-1} = \bar{\phi}$
\end{proof}

%\begin{Prop}\label{prop:leapfrog}
%  Soient $\epsilon > 0, L \in \nset^*$ et $(x_0, p_0) \in \rset^{d} \times \rset^{d}$. Alors les positions et quantité de mouvement obtenues par l'algorithme leapfrog vérifient
%  \begin{align*}
%    x_L &= x_0 + \epsilon \sum_{i=0}^{L-1} \nabla K \left( p_i - \frac{\epsilon}{2} \nabla U(x_i) \right) \\
%    p_L &= p_0 - \frac{\epsilon}{2} \nabla U(x_0) - \epsilon \sum_{i=1}^{L-1} \nabla U(x_i) - \frac{\epsilon}{2} \nabla U(x_L) 
%  \end{align*}
%  %et dans le cas particulier où $K : p \mapsto \frac{p^\top p }{2}$
%\end{Prop}      

\section{Hamiltonian Monte-Carlo}

\subsection{Cas idéal : solution exacte de la dynamique hamiltonienne}

Dans le cas idéal où l'on a accès à une solution exacte de la dynamique hamiltonienne, i.e. si l'on connait le flot $\phi_t$ pour tout $t \in \rset_+$, la stratégie proposée est de suivre le flot à chaque étape après avoir tiré aléatoirement une quantité de mouvement.

\begin{center}
	\begin{algorithm}[H]
		\KwData{$h_\pi$ proportionnel à la densité cible, $t$ une durée sur laquelle suivre la dynamique}
		$X_0 \leftarrow x \in \xset$ arbitraire\;
		\Repeat{une condition d'arrêt}{
			$\tilde{P}_k \sim \mathcal \nu$ et $\tilde{P}_k \indep (Z_0, \cdots, Z_{k})$ \tcp*{Tirer la quantité de mouvement}
			$\tilde{Z}_k \leftarrow (X_k, \tilde{P}_k)$ \;
			$Z_{k+1} = (X_{k+1}, P_{k+1}) \leftarrow \phi_t(\tilde{Z}_k)$ \tcp*{Suivre la dynamique}
		}
		\KwRet{$(X_k)_k$}
		\caption{Hamiltonian Monte-Carlo, cas idéal}
		\label{algo:HMC-ideal}
	\end{algorithm}
\end{center}

\begin{Def}[semi-groupe Markovien]
	Une famille $(P_t)_{t \in \rset_+}$ de noyaux de Markov est un semi-groupe Markovien si $\forall t,s \in \rset_+, P_{t+s} = P_t P_s$
\end{Def}

\begin{Prop}[invariance de la loi]\label{prop:inv-ideal}
	On se place sous les hypothèses \eqref{eq:hyp}. Alors

	\begin{enumerate}
		\item Pour $t \in \rset_+$, le processus $(Z_k)_{k \in \nset}$ défini par l'algorithme \ref{algo:HMC-ideal} est une chaîne de Markov homogène de noyau $QP_t$ sur $\rset^{2d}$ avec
		% $$
		% Q : \fundef{
		% \rset^{2d} \times \calB(\rset^{2d}) & \to & [0,1] \\
		% ((x,p), A) & \mapsto & \delta_x \otimes \nu(A)
		% }
		% $$
		% $$
		% P_t : \fundef{
		% \rset^{2d} \times \calB(\rset^{2d}) & \to & [0,1] \\
		% (z, A) & \mapsto & \delta_{\phi_t(z)}(A) = \ind_A(\phi_t(z))
		% }
		% $$
		$$
		Q((x,p),\cdot) = \delta_x \otimes \nu
		$$
		$$
		P_t((x,p),\cdot) = \delta_{\phi_t(z)}
		$$
		\item $(P_t)_{t \in \rset_+}$ est un semi-groupe Markovien.
		\item Pour tout $t \in \rset_+$, $\widetilde{\pi}$ est $QP_t$-invariante. 
	\end{enumerate}
\end{Prop}
\begin{proof}
	\begin{enumerate}
		\item On note $(\calF_k)_{k \in \nset} = (\sigma(Z_j, j\leq k))_{k \in \nset}$ la filtration canonique associée à $Z$. Soit $A \in \calB(\rset^{2d})$ alors $\PP[Z_{k+1} \in A \given{\calF_k}] = \PP[\phi_t(X_k, \tilde{P}_k) \in A \given{\calF_k}] = \PP[\phi_t(X_k, \tilde{P}_k) \in A \given{\calF_k}]$. Comme $\tilde{P}_k \indep \calF_k$, on a donc $\PP[Z_{k+1} \in A \given{\calF_k}] = \hat{F}(X_k)$ où $\forall x \in \rset^d, \hat{F}(x) = \PP[\phi_t(x, \tilde{P}_1) \in A]$. Ainsi $(Z_k)_{k \in \nset}$ est une chaîne de Markov homogène de noyau $N$ tel que
		$$
		\forall (x,p) \in \rset^{2d}, \forall A \in \calB(\rset^{2d}), N((x,p), A) = \PP[\phi_t(x, \tilde{P}_1) \in A] = \int \ind_A(\phi_t(x, p_1)) \nu(dp_1).
		$$
		Or
		$$
		Q P_t((x,p),A) = \int Q((x,p), dx_1dp_1) P_r((x_1, p_1), A) = \int \delta_x(dx_1) \nu(dp_1) \ind_A(\phi_t(x_1, p_1)) = \int \ind_A(\phi_t(x,p_1)) \nu(dp_1).
		$$
		Donc $(Z_k)_{k \in \nset}$ est une chaîne de Markov homogène de noyau $QP_t$.

		\item Soient $t,s \in \rset_+$, $z \in \rset^{2d}$ et $A \in \calB(\rset^{2d})$. Alors
		$$
		P_t P_s (z, A) = \int P_t(z, dz_1) P_s(z_1, A) = \int \delta_{\phi_t(z)}(dz_1) \ind_A(\phi_s(z_1)) = \ind_A(\phi_s \circ \phi_t(z)) = \ind_A(\phi_{s+t}(z)) = P_{s+t} = P_{t+s}
		$$

		\item Montrons que $\widetilde{\pi}$ est $Q$ et $P_t$ invariante (elle sera donc $QP_t$-invariante). Soit $A \in \calB(\rset^{2d})$, alors
		$$
		\widetilde{\pi} Q (A) = \int \pi(dx) \nu(dp) \delta_x(dx_1) \nu(dp_1) \ind_A(x_1, p_1) = \int \pi(dx) \nu(dp_1) \ind_A(x, p_1) = \int \widetilde{\pi}(dxdp_1) \ind_A(x, p_1) = \widetilde{\pi}(A) 
		$$

		Soient $t \in \rset_+$ et $A \in \calB(\rset^{2d})$, alors
		\begin{align*}
		\widetilde{\pi} P_t (A)
		&= \int \widetilde{\pi}(dz) P_t(z, A) \\
		&\propto \int \exp \left(-\frac{H(z)}{T} \right) \ind_A(\phi_t(z)) dz\\
		&\overset{(*)}{=} \int \exp \left(-\frac{H \circ \phi_t^{-1}(y)}{T} \right) \ind_A(y) dy \\
		&\overset{(**)}{=} \int \exp \left(-\frac{H(y)}{T} \right) \ind_A(y) dy \\
		&= \widetilde{\pi}(A).
		\end{align*}
		$(*)$ : en faisant le changement de variable $y = \phi_t(z)$ comme $\phi_t$ est un $\calC^1$ difféomorphisme (proposition \ref{prop:rev}) de jacobien égal à $1$ (proposition \ref{prop:vol}) \\
		$(**)$ : par la proposition \ref{prop:conservation} on sait que $H \circ \phi_t = H$. De plus sous les hypothèses \eqref{eq:hyp}, $\bar{H} = H \circ s = H$ où $s : (x,p) \mapsto (x, -p)$. Ainsi $H \circ \phi_t^{-1} = H \circ s \circ \phi_t \circ s = H \circ \phi_t \circ s = H \circ s = H$ 
	\end{enumerate}
\end{proof}


\subsection{Cas concret : solution approchée de la dynamique hamiltonienne}
Dans le cas où l'on n'a pas accès aux solutions exactes de \eqref{eq:edo}, on utilise le flot appoché $\hat{\phi}_L$. La première idée serait de simplement remplacer $\phi_t$ par $g \circ \hat{\phi}_L$ dans l'algorithme \ref{algo:HMC-ideal} où $g : \rset^{2d} \to \rset^{2d}$ est une fonction à déterminer. On obtient alors une chaîne de Markov de noyau $Q\hat{P}_L$ avec $Q$ défini en proposition \ref{prop:inv-ideal} et
$$
\hat{P}_L : \fundef{
  \rset^{2d} \times \calB(\rset^{2d}) & \to & [0,1] \\
  (z, A) & \mapsto & \delta_{g \circ \hat{\phi}_L(z)}(A)
}
$$
$\widetilde{\pi}$ est toujours $Q$-invariante mais n'est plus $\hat{P}_L$-invariante car on n'a plus la propriété de conservation du hamiltonien. Néanmoins la proposition suivante montre que pour un bon choix de $g$, il est possible d'utiliser le mouvement proposé dans un algorithme de Métropolis.

\begin{Prop}[symétrie]
  Sous les hypothèses \eqref{eq:hyp}, en prenant $g = s : (x, p) \mapsto (x,-p)$, alors pour $L \in \nset^*$, $\hat{P}_L$ est un noyau à densité $\hat{p}_L$ par rapport à la mesure de Lebesgue et $\hat{p}_L$ est symétrique i.e $\forall z,z' \in \rset^{2d}, \hat{p}_L (z, z') = \hat{p}_L (z',z)$. 
\end{Prop}
\begin{proof}
  Il est facile de voir que $\hat{p}_L : (z,z') \mapsto \delta_{s \circ \hat{\phi}_L(z), z'}$. Ainsi $\hat{p}_L(z,z') =  \delta_{s \circ \hat{\phi}_L(z), z'} = \delta_{(s \circ \hat{\phi}_L)^{-1}(z'), z} = \delta_{\hat{\phi}_L(z'), z} = \hat{p}_L(z',z)$ car, d'après la proposition, \ref{prop:rev-leapfrog} $(s\circ \hat{\phi}_L)^{-1} = s \circ \hat{\phi}_L$. 
\end{proof}

\begin{center}
	\begin{algorithm}[H]
		\KwData{$h_\pi$ proportionnel à la densité cible, $\epsilon$ pas du saute-mouton, $L$ nombre de pas du saute-mouton}
		$X_0 \leftarrow x \in \xset$ arbitraire\;
		\Repeat{une condition d'arrêt}{
			$P_k \sim \mathcal N(0,1)$\tcp*{Tirer la quantité de mouvement}
			$(X_{prop},P_{prop}) \leftarrow \texttt{leapfrog}(X_k,P_k)$\tcp*{Proposer un mouvement}
			$U_k \leftarrow U(X_k)$\;
			$K_k \leftarrow \norm{P_k}^2/2$\;
			$U_{prop} \leftarrow U(X_{prop})$\;
			$K_{prop} \leftarrow \norm{P_{prop}}^2/2$\;
			\eIf{$\mathcal U([0,1]) < \exp(U_k-U_{prop}+K_k-K_{prop})$}{
				$X_{k+1} \leftarrow X_{prop}$ \tcp*{Accepter}
			}{
				$X_{k+1} \leftarrow X_k$ \tcp*{Rejeter}
			}
		}
		\KwRet{$(X_k)_k$}
		\caption{Hamiltonian Monte-Carlo}
		\label{algo:HMC}
	\end{algorithm}
\end{center}


\section{Simulations}

% Rassembler les parties précédentes pour écrire tout l'algorithme HMC, et présenter/commenter les résultats des simulations


\pagebreak
\bibliographystyle{plain}
\bibliography{ref}
\end{document}


