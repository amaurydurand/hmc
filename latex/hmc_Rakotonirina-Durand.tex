\documentclass[10pt,a4paper]{article}
\usepackage{init}
\usepackage{defs}
\usepackage{color}
\newcommand{\red}{\color{red}}
\definecolor{gray}{rgb}{0.4, 0.4, 0.4}

\title{Introduction aux méthodes de Monte Carlo par dynamique Hamiltonienne}
\author{Shmuel RAKOTONIRINA-RICQUEBOURG, Amaury DURAND}
\begin{document}
\maketitle
\tableofcontents
\section{Introduction}
Ce rapport présente le travail effectué lors d'un projet du cours d'approfondissements en chaînes de Markov par Eric Moulines dans le cadre du master Mathématiques de l'aléatoires à l'université Paris-Sud. Le contenu présenté ci-dessous repose essentiellement sur \cite{Neal-hmc} et \cite{Douc-mc}. 

Les méthodes de Monte Carlo par dynamique Hamiltonienne, plus communément appelées Hamiltonian Monte Carlo (HMC), font partie d'une grande famille de méthodes de simulation : les méthodes de Monte Carlo, et plus précisemment dans la famille des méthodes de Monte Carlo par chaînes de Markov (ou Monte Carlo Markov Chains). Ces méthodes se placent dans le cadre suivant :
\begin{Def}[Carde général]
  Soit $(\xset, \calX)$ un espace mesurable. Soit $\pi$ une loi de probabilité sur cet espace. On suppose que $\pi$ n'est connue qu'à un facteur de proportionnalité près i.e on connait $\lambda \pi$ où $\lambda \in \rset$ une constante. Le but est de pouvoir approcher $\pi f = \EE[f(X)]$ avec $X \sim \pi$ et $f \in \fset_+(\xset, \calX) \cup \fset_b(\xset, \calX)$.
\end{Def}

\subsection{Méthodes de Monte Carlo}

La méthode de Monte Carlo la plus simple (appelée Monte Carlo naïf) est la suivante : supposons que l'on sait simuler des variables aléatoires de loi $\pi$ alors en tirant $n$ échantillons $X_1, \cdots, X_n \siid \pi$, la loi des grands nombre nous indique qu'une bonne approximation de $\pi f$ est $\frac{1}{n} \sum_{i=1}^n f(X_i)$. D'autres méthodes permettent d'atteindre le même but en ne sachant pas simuler de variables aléatoires de loi $\pi$. C'est le cas par exemple de l'échantillonnage d'importance qui consiste à simuler des variables i.i.d sous une loi différente de $\pi$.

Dans ces deux cas, l'approximation de $\pi f$ repose sur une simulation de variables aléatoires i.i.d. Cette particularité permet alors de montrer des résultats de convergence notamment grâce à la loi des grands nombre. Les méthodes de Monte Carlo par chaînes de Markov ne reposent pas sur le caractère i.i.d des variables mais sur les propriétés des Chaînes de Markov. 

\subsection{Monte Carlo Markov Chains (MCMC)}
Les méthode MCMC se basent sur les notions de loi invariante, de réversibilité et d'ergodicité.
\subsubsection{Loi invariante et réversibilité}
On considère $P$ un noyau de Markov sur $\xset \times \calX$.
\begin{Def}
  Une loi de probabilité $\pi$ sur $(\xset, \calX)$ est dite
  \begin{itemize}
  \item $P$-invariante si $\pi P = \pi$
  \item $P$-réversible si $\forall A,B \in \calX$, $\pi \otimes P (A \times B) = \pi \otimes P (B \times A)$
\end{itemize}
\end{Def}

\begin{Prop}
  Toute probabilité $P$-réversible est $P$-invariante.
  % Soit $\pi$ une loi de probabilité sur $(\xset, \calX)$ alors 
  % $$
  % \pi \text{ est } P\text{-reversible} \Rightarrow \pi \text{ est } P\text{-invariante}
  % $$
\end{Prop}

\subsubsection{Ergodicité}
% \begin{Def}[Système ergodique]
%   Soit $(\Omega, \calB, \PP)$ un espace de probabilité. Soit $T : (\Omega, \calB) \to (\Omega, \calB)$ une application mesurable. 
%   \begin{itemize}
%   \item On dit que $\PP$ est invariante pour $T$ si $\forall A \in \calB, \PP[T^{-1}(A)] = \PP[A]$. Dans ce cas, on dit que $(\Omega, \calB, \PP, T)$ est un système dynamique.
%   \item $A \in \calB$ est dit invariant pour $T$ si $A = T^{-1}(A)$.
%   \item Si pour tout $A$ invariant pour $T$ on a $\PP[A] \in \ens{0,1}$ alors on dit que $(\Omega, \calB, \PP, T)$ est un système ergodique. 
%   \end{itemize}
% \end{Def}

% \begin{Thm}\label{thm:ergodic}
%   Soit $P$ un noyau de Markov sur $\xset \times \calX$ et on se place sur l'espace canonique $(\xset^n, \calX^{\otimes n})$. On note $ \theta : \fundef{ \xset^\nset &\to& \xset^\nset \\ (\omega_t)_{t\in \nset} &\mapsto& (\omega_{t+1})_{t \in \nset}}$ et  $\forall k \in \nset^*, \theta_k = \theta_{k-1} \circ \theta$ avec $\theta_0 = Id$. On suppose
%   \begin{enumerate}
%   \item $P$ possède une loi invariante $\pi$
%   \item $(\xset^\nset, \calX^{\otimes \nset}, \PP_\pi, \theta)$ est ergodique
%   \end{enumerate}
%   Alors pour toute v.a $Y \in L^1(\xset^\nset, \calX^{\otimes \nset}, \PP_\pi)$, pour $\pi$-presque tout $x \in \xset$, 
% $$
%   \frac{1}{n} \sum_{k=0}^{n-1} Y \circ \theta_k \xrightarrow[n \to +\infty]{\PP_x\text{-}\ps} \EE{\pi}[Y]
%   $$
% \end{Thm}

% \begin{Rque}
%   Considérons $(X_k)_{k \in \nset}$ la chaîne de Markov canonique de noyau $P$ et $f \in \fset_+(\xset, \calX) \cup \fset_b(\xset, \calX)$ alors en prenant $Y = f(X_0)$, on a $Y \circ \theta_k = f(X_k)$ et $\EE{\pi}[Y] = \EE{\pi}[f(X_0)] = \pi f$ donc le résultat du théorème \ref{thm:ergodic} se réécrit : pour $\pi$-presque tout $x \in \xset$, 
% $$
%   \frac{1}{n} \sum_{k=0}^{n-1} f(X_k) \xrightarrow[n \to +\infty]{\PP_x\text{-}\ps} \pi f
%   $$
%   Ce qui montre que l'on peut avoir une bonne approximation de $\pi f$ en construisant une chaîne de Markov. De plus il est intéressant de constater que la convergence est $\PP_x$ presque sûre pour $\pi$-presque tout $x \in \xset$ ce qui signifie que quelque soit le point de départ, on est sûr d'avoir une bonne approximation de $\pi f$ si on attend suffisamment longtemps.
% \end{Rque}

\begin{Thm}\label{thm:ergodic}
  Soit $(X_k)_{k \in \nset}$ une chaîne de Markov de noyau $P$ admettant une loi invariante $\pi$. Alors pour tout $f \in \fset_+(\xset, \calX) \cup \fset_b(\xset, \calX)$ et pour $\pi$-presque tout $x \in \xset$,
  $$\frac{1}{n} \sum_{k=0}^{n-1} f(X_k) \xrightarrow[n \to +\infty]{\PP_x\text{-}\ps} \pi f$$
\end{Thm}

Les algorithmes MCMC visent à construire, à partir de $\pi$, une chaîne de Markov vérifiant les hypothèses de ce théorème afin d'approcher $\pi f$ par $\frac{1}{n} \sum_{k=0}^{n-1} f(X_k)$.

% \subsubsection{Algorithme de Metropolis (Random Walk Metropolis)}

% Avant de présenter la méthode HMC, nous définissons ici l'algorithme de Metropolis que nous utiliserons comme base de comparaison. On se place dans le cas où $\xset = \rset^d$ et $\calX = \calB(\rset^d)$. 

% On suppose que $\pi$ a une densité $h_\pi$ par rapport à une mesure $\mu$. On considère de plus un loi $Q$ (appelée loi instrumentale) sur $(\xset, \calX)$ de densité $q$ par rapport à $\mu$ telle que $\forall x \in \xset, q(x) = q(-x) $. La construction de la chaîne de Markov se fait en proposant un mouvement (dont l'incrément suit $Q$), puis en acceptant ou en rejetant ce mouvement (algorithme \ref{algo:metropolis}).

% \begin{center}
% \begin{algorithm}[H]
%   $X_0 \leftarrow x \in \xset$ arbitraire\;
%   $(U_k)_{k \in \nset} \siid Q$ \;
%   \Repeat{une condition d'arrêt}{
%     $Y_{k+1} \leftarrow X_k + U_{k+1}$ \tcc*{Proposer un mouvement}
%     $\alpha_{k+1} \leftarrow \alpha(X_k, Y_{k+1})$ où $\alpha(x,y) = 1 \wedge \frac{h_{\pi}(y)}{h_\pi(x)}$\;
%     $X_{k+1} \leftarrow \piecewise{ Y_{k+1} & \text{avec probabilité } \alpha_{k+1} \\ X_k & \text{with probability } 1 - \alpha_{k+1}}$ \tcc*{Accepter ou rejeter le mouvement}
%     }
%   \caption{Random Walk Metropolis}
%   \label{algo:metropolis}
% \end{algorithm}
% \end{center}

% Dans cet algorithme, on cherche à explorer l'espace $\xset$ en visitant moins souvent les régions où $h_\pi$ est faible (peu chargées par $\pi$) car ce sont les régions de forte probabilité qui donnent des informations sur la loi $\pi$. Ainsi, si le mouvement proposé vérifie $h_\pi(Y_{k+1}) \geq h_\pi(X_k)$, on se dirige vers une zone plus chargée par $\pi$, on accepte donc le mouvement. Dans le cas contraire, il est moins intéressant de bouger. On autorise quand même le mouvement avec une probabilité $\frac{h_\pi(Y_{k+1})}{h_\pi(X_k)}$ d'autant plus faible que la position proposée est dans une région de probabilité faible.

% % Une explication intuitive de cet algorithme est de voir que l'on cherche à visiter l'espace $\xset$ sans pour autant aller dans des régions où $h_\pi$ est faible (et donc des régions de probabilité faible) car ce sont les régions de forte probabilité qui donnent des informations sur la loi $\pi$. Ainsi si le mouvement proposé est tel que $h_\pi(Y_{k+1}) \geq h_\pi(X_k)$ on accepte le mouvement avec probabilité $1$. Dans le cas contraire on considère qu'il n'est pas très intéressant de bouger. Néanmoins on autorise quand même un mouvement avec une probabilité de $\frac{h_\pi(Y_{k+1})}{h_\pi(X_k)}$ qui sera d'autant plus faible que la position proposée est dans une région de probabilité faible.

\subsubsection{Algorithme de Metropolis-Hastings}

L'algorithme HMC est une version améliorée de l'algorithme de Metropolis-Hastings. Nous allons donc d'abord définir ce dernier et nous l'utiliserons plus tard comme base de comparaison. On se place dans le cas où $\xset = \rset^d$ et $\calX = \calB(\rset^d)$. 

On suppose que $\pi$ a une densité $h_\pi$ par rapport à une mesure $\mu$. On considère de plus un noyau markovien $Q$ sur $(\xset, \calX)$ de densité $q$ par rapport à $\mu$. $Q$ est appelé noyau instrumental. La construction de la chaîne de Markov se fait en proposant un mouvement via $Q$, puis en acceptant ou en rejetant ce mouvement (algorithme \ref{algo:metropolis-hastings}).

\begin{center}
\begin{algorithm}[H]
  $X_0 \leftarrow x \in \xset$ arbitraire\;
  \Repeat{une condition d'arrêt}{
    $Y_{k+1} \sim Q(X_k,\cdot)$ \tcc*{Proposer un mouvement}
    $\alpha_{k+1} \leftarrow \alpha(X_k, Y_{k+1})$ où $\alpha(x,y) = 1 \wedge \frac{h_{\pi}(y)q(y,x)}{h_\pi(x)q(x,y)}$\;
    $X_{k+1} \leftarrow \piecewise{ Y_{k+1} & \text{avec probabilité } \alpha_{k+1} \\ X_k & \text{with probability } 1 - \alpha_{k+1}}$ \tcc*{Accepter ou rejeter le mouvement}
    }
  \caption{Metropolis-Hastings}
  \label{algo:metropolis-hastings}
\end{algorithm}
\end{center}

Dans cet algorithme, on cherche à explorer l'espace $\xset$ en visitant moins souvent les régions où $h_\pi$ est faible (peu chargées par $\pi$) car ce sont les régions de forte probabilité qui donnent des informations sur la loi $\pi$. Ainsi, si le mouvement proposé vérifie $h_\pi(Y_{k+1}) \geq h_\pi(X_k)$, on se dirige vers une zone plus chargée par $\pi$, on accepte donc le mouvement. Dans le cas contraire, il est moins intéressant de bouger. On autorise quand même le mouvement avec une probabilité $\frac{h_\pi(Y_{k+1})}{h_\pi(X_k)}$ d'autant plus faible que la position proposée est dans une région de probabilité faible.

\begin{Rque}
  Il y a d'autres fonctions de rejet $\alpha$ qui permettent à l'algorithme de fonctionner. De même, le choix du noyau $Q$ est un degré de liberté de l'algorithme.

  Le choix de $Q$ est en fait une difficulté de l'algorithme, puisque le mouvement proposé doit permettre l'exploration de tout l'espace (afin de passer régulièrement par toutes les régions fortement chargées par $\pi$). Nous verrons que l'algorithme HMC résout ce problème en proposant le mouvement selon une \emph{dynamique hamiltonienne}.
\end{Rque}

\begin{Def}
  On parle de \emph{Random-walk Metropolis} quand $Q$ est le noyau d'une marche aléatoire
  $$Y_{k+1} \leftarrow X_k + U_{k+1}$$
  avec $(U_k)_{k \in \nset}$ iid. On supposera que $Q$ admet une densité $q$ par rapport à $\mu$ avec $q(x) = q(-x)$.
\end{Def}

\begin{center}
\begin{algorithm}[H]
  $X_0 \leftarrow x \in \xset$ arbitraire\;
  (U_k)_{k \in \nset} \siid Q$ \;
  \Repeat{une condition d'arrêt}{
    $Y_{k+1} \sim X_k + U_{k+1}$ \tcc*{Proposer un mouvement}
    $\alpha_{k+1} \leftarrow \alpha(X_k, Y_{k+1})$ où $\alpha(x,y) = 1 \wedge \frac{h_{\pi}(y)}{h_\pi(x)}$\;
    $X_{k+1} \leftarrow \piecewise{ Y_{k+1} & \text{avec probabilité } \alpha_{k+1} \\ X_k & \text{with probability } 1 - \alpha_{k+1}}$ \tcc*{Accepter ou rejeter le mouvement}
    }
  \caption{Random Walk Metropolis}
  \label{algo:metropolis}
\end{algorithm}
\end{center}

\section{Hamiltonian Monte Carlo}

On se place dans le cas où $\xset = \rset^d$, $\calX = \calB(\rset^d)$ et $\mu$ est la mesure de Lebesgue. 

L'algorithme HMC fait partie des algorithmes MCMC. Il diffère de l'algorithme de Metropolis-Hastings en ce que le mouvement proposé ne sera pas tiré selon une chaîne de Markov de noyau $Q$ mais selon une \emph{dynamique hamiltonienne} associée à la loi à simuler.

\subsection{Principe général}

La dynamique hamiltonienne décrit le mouvement d'un objet qui glisse sans frottement le long d'une surface ou d'une courbe. L'objet est décrit par sa position $x \in \rset^d$ et sa vitesse $v \in \rset^d$ et on lui associe des énergies potentielle $U(x)$ et cinétique $K(v)$. En physique, une énergie $E$ et une température $T$ est associée à une loi de probabilité par la loi de Boltzmann
$$\forall x \in \rset^d,  p(x) \propto \exp \left( -\frac{E(x)}{T} \right)$$
Ici on choisit donc l'énergie potentielle pour avoir
\begin{equation}\label{eq:canonical-dist}
\forall x \in \rset^d,  h_\pi(x) \propto \exp \left( -\frac{U(x)}{T} \right)
\end{equation}

En dimension 1, on décrit le glissement sans frottement d'un objet sur une rampe. Comme l'énergie potentielle est proportionnelle à là hauteur de la rampe, \eqref{eq:canonical-dist} donne que les creux de la rampe représentent les régions de forte probabilité. L'objet aura tendance à glisser vers les creux de la rampe, mais peut remonter une pente si sa vitesse $v$ (et donc son énergie cinétique $K(v)$ est assez grande.

L'algorithme HMC propose ses mouvements selon le principe suivant
\begin{enumerate}
  \item Considérer un objet placé en $x = X_k$ de vitesse aléatoire $v$ (de loi à préciser)
  \item Simulation la dynamique hamiltonienne (déterministe) de cet objet pendant une durée fixée, pour l'énergie potentielle $U$ donnée par \ref{eq:canonical-dist}
  \item Considérer $Y_{k+1}$ la position de cet objet à la fin de la simulation, et accepter ou rejeter le mouvement selon une fonction de rejet $\alpha$ dépendant de $U$
\end{enumerate}

\subsection{Outils nécessaires}

% On a besoin
% % d'une simulation de dynamique hamiltonienne
% % que l'algorithme ci-dessus "conserve la distribution"

\subsection{Propriétés de l'algorithme}

% Lister les propriétés et bénéfices de HMC en admettant qu'on a les outils de la subsection précédente

\section{Dynamique hamiltonienne}

% Donner rapidement les résultats déterministes de l'article sur l'hamiltonnien (2.1 et 2.2 de l'article), peut-être qu'on peut se contenter du cas U = -log(h_\pi) et K = \norm{v}/2 (T = 1, m = 1)
% Expliquer la méthode du saute-mouton pour simuler la dynamique hamiltonienne, en quoi elle est meilleure, en quoi elle peute-être simplifiée (2.3 de l'article), éventuellement en quoi elle "préserve le volume"

\section{Algorithme et simulations}

% Rassembler les parties précédentes pour écrire tout l'algorithme HMC, et présenter/commenter les résultats des simulations

\pagebreak
\bibliographystyle{plain}
\bibliography{ref}
\end{document}


