\documentclass[10pt]{beamer}
%%%%%%%%%%%%%%%%%%%%%%%%%%%%%%%%%%%%%%%%%%%%%%%%%%%%%%%%%%%
%%%%% 	French, accents and font coding     	      %%%%%
%%%%%%%%%%%%%%%%%%%%%%%%%%%%%%%%%%%%%%%%%%%%%%%%%%%%%%%%%%%

\usepackage[french]{babel}
%\usepackage{enumitem} 
\usepackage{fancyhdr}
\usepackage[utf8]{inputenc}
\usepackage{bbm, amsmath, amssymb, amsthm, amsfonts}			% for theorem definitions


\usepackage{graphicx}			% for images and graphics
\usepackage{stmaryrd}			% for \llbracket symbols [[ ]]

%%%%%%%%%%%%%%%%%%%%%%%%%%%%%%%%%%%%%%%%%%%%%%%%%%%%%%%%%%%
%%%%%  		 Theorems etc.                        %%%%%
%%%%%%%%%%%%%%%%%%%%%%%%%%%%%%%%%%%%%%%%%%%%%%%%%%%%%%%%%%%

\newtheorem{Thm}{Théorème}[section]
\newtheorem{Cor}{Corolaire}[Thm]
\newtheorem{Prop}{Proposition}[section]
\newtheorem{Lem}{Lemme}[section]
\theoremstyle{definition}
\newtheorem{Def}{Définition}[section]
\newtheorem{Ex}{Exemple}[section]
\newtheorem{Exo}{Exercice}[section]
\theoremstyle{remark}
\newtheorem*{Rque}{Remarque}
\theoremstyle{definition}
\newtheorem{Algo}{Algorithme}[section]

%%%%%%%%%%%%%%%%%%%%%%%%%%%%%%%%%%%%%%%%%%%%%%%%%%%%%%%%%%%
%%%%%  		Hyperlinks                            %%%%%
%%%%%%%%%%%%%%%%%%%%%%%%%%%%%%%%%%%%%%%%%%%%%%%%%%%%%%%%%%%
\usepackage{hyperref}
\hypersetup{
    colorlinks=true,
    linkcolor=blue,
    filecolor=magenta,      
    urlcolor=cyan,
}
%%%%%%%%%%%%%%%%%%%%%%%%%%%%%%%%%%%%%%%%%%%%%%%%%%%%%%%%%
%%%%		For Algorithms
%%%%
%%%%%%%%%%%%%%%%%%%%%%%%%%%%%%%%%%%%%%%%%%%%%%%%%%%%%%%%%
\usepackage[ruled, vlined, french, onelanguage, algosection, nofillcomment, scleft]{algorithm2e}
\SetKwBlock{KwInit}{Initialization :}{endInit}


%%%%%%%%%%%%%%%%%%%%%%%%%%%%%%%%%%%%%%%%%%%%%%%%%%%%%%%%%%%
%%%%%  		Code			 %%%%%
%%%%%%%%%%%%%%%%%%%%%%%%%%%%%%%%%%%%%%%%%%%%%%%%%%%%%%%%%%%
\usepackage[section]{minted}
\usemintedstyle{borland}
\usepackage{color}
\definecolor{bg}{rgb}{0.95,0.95,0.95}

%%%%%%%%%%%%%%%%%%%%%%%%%%%%%%%%%%%%%%%%%%%%%%%%%%%%%%%%%
%%%%%		table of contents						%%%%%
%%%%%%%%%%%%%%%%%%%%%%%%%%%%%%%%%%%%%%%%%%%%%%%%%%%%%%%%%
\AtBeginSection{
	\begin{frame}
		\tableofcontents[currentsection, hideothersubsections]
	\end{frame}
}

%%%%%%%%%%%%%%%%%%%%%%%%%%%%%%%%%%%%%%%%%%%%%%%%%%%%%%%%%%%
%%%%%  		 latin abbreviations                  %%%%%
%%%%%%%%%%%%%%%%%%%%%%%%%%%%%%%%%%%%%%%%%%%%%%%%%%%%%%%%%%%
\newcommand{\ie}{{\em i.e.,~}}
\newcommand{\eg}{{\em e.g.,~}}
\newcommand{\lcf}{{\em cf.~}}



%%%%%%%%%%%%%%%%%%%%%%%%%%%%%%%%%%%%%%%%%%%%%%%%%%%%%%%%%%%
%%%%%  		 Style math                         %%%%%
%%%%%%%%%%%%%%%%%%%%%%%%%%%%%%%%%%%%%%%%%%%%%%%%%%%%%%%%%%%

\newcommand{\nset}{\mathbb{N}}
\newcommand{\zset}{\mathbb{Z}}
\newcommand{\rset}{\mathbb{R}}
\newcommand{\cset}{\mathbb{C}}
\newcommand{\qset}{\mathbb{Q}}
\newcommand{\kset}{\mathbb{K}}
\newcommand{\tset}{\mathbb{T}}
\newcommand{\xset}{\mathbb{X}}
\newcommand{\yset}{\mathbb{Y}}
\newcommand{\uset}{\mathbb{U}}
\newcommand{\fset}{\mathbb{F}}
\newcommand{\mset}{\mathbb{M}}

\newcommand{\ps}{\text{ p.s. }}
\newcommand{\geqrespleq}{\underset{\mbox{ (resp $\leq$) }}{\geq}}
\newcommand{\leqrespgeq}{\underset{\mbox{ (resp $\geq$) }}{\leq}}



\usepackage{xifthen}
%\newcommand{\EE}[1][]{%
%  \ifthenelse{\isempty{#1}}%
%    {\mathbb{E}}% if #1 is empty
%    {\mathbb{E}\left[#1\right]}% if #1 is not empty
%}

%\newcommand{\PP}[1][]{%
%  \ifthenelse{\isempty{#1}}%
%    {\mathbb{P}}% if #1 is empty
%    {\mathbb{P}\left(#1\right)}% if #1 is not empty
%}

\newcommand{\law}[1][]{%
  \ifthenelse{\isempty{#1}}%
    {\calL}% if #1 is empty
    {\calL \left(#1\right)}% if #1 is not empty
}

%\newcommand{\pp}[1][]{%
%  \ifthenelse{\isempty{#1}}%
%    {p}% if #1 is empty
%    {p\left(#1\right)}% if #1 is not empty
%}
% 
%\newcommand{\Cov}[1][]{%
%  \ifthenelse{\isempty{#1}}%
%    {\mathrm{Cov}}% if #1 is empty
%    {\mathrm{Cov}\left(#1\right)}% if #1 is not empty
%}
% 
%\newcommand{\Var}[1][]{%
%  \ifthenelse{\isempty{#1}}%
%    {\mathrm{Var}}% if #1 is empty
%    {\mathrm{Var}\left(#1\right)}% if #1 is not empty
%  }

\usepackage{xparse}
\DeclareDocumentCommand{\PP}{go}
{\IfNoValueTF{#2}
  {\IfNoValueTF{#1}
    {\mathbb{P}}
    {\mathbb{P}_{#1}}%
  }
  {\IfNoValueTF{#1}
    {\mathbb{P}\left(#2\right)}
    {\mathbb{P}_{#1}\left(#2\right)}%
  }%
}

\DeclareDocumentCommand{\EE}{go}
{\IfNoValueTF{#2}
  {\IfNoValueTF{#1}
    {\mathbb{E}}
    {\mathbb{E}_{#1}}%
  }
  {\IfNoValueTF{#1}
    {\mathbb{E}\left[#2\right]}
    {\mathbb{E}_{#1}\left[#2\right]}%
  }%
}

\DeclareDocumentCommand{\pp}{go}
{\IfNoValueTF{#2}
  {\IfNoValueTF{#1}
    {p}
    {p_{#1}}%
  }
  {\IfNoValueTF{#1}
    {p\left(#2\right)}
    {p_{#1}\left(#2\right)}%
  }%
}

\DeclareDocumentCommand{\Cov}{go}
{\IfNoValueTF{#2}
  {\IfNoValueTF{#1}
    {\mathrm{Cov}}
    {\mathrm{Cov}_{#1}}%
  }
  {\IfNoValueTF{#1}
    {\mathrm{Cov}\left(#2\right)}
    {\mathrm{Cov}_{#1}\left(#2\right)}%
  }%
}

\DeclareDocumentCommand{\Var}{go}
{\IfNoValueTF{#2}
  {\IfNoValueTF{#1}
    {\mathrm{Var}}
    {\mathrm{Var}_{#1}}%
  }
  {\IfNoValueTF{#1}
    {\mathrm{Var}\left(#2\right)}
    {\mathrm{Var}_{#1}\left(#2\right)}%
  }%
}



\def\1{\mathbf 1}
\def\ind{\mathbbm 1}
\newcommand{\mb}{\mathbf}
\newcommand{\ud}{\,\mathrm{d}} 
\newcommand{\given}[1][{}]{\;\middle\vert\;{#1} }
\def\indep{\perp\!\!\!\perp}


%%%%%%%%%%%%%%%%%%%%%%%%%%%%%%%%%%%%%%%%%%%%%%%%%%%%%%%%%%%
%%%%%        equality symbols                        %%%%%
%%%%%%%%%%%%%%%%%%%%%%%%%%%%%%%%%%%%%%%%%%%%%%%%%%%%%%%%%%%
\newcommand{\deq}{\stackrel{\rm def}{=}}
\newcommand{\eqlaw}{\stackrel{\rm loi}{=}}
\newcommand{\siid}{\stackrel{\rm iid}{\sim}}
\newcommand{\cps}{\xrightarrow[n \to +\infty]{\ps}}
\newcommand{\cprob}{\xrightarrow[n \to +\infty]{P}}
\newcommand{\claw}{\xrightarrow[n \to +\infty]{\calL}}


%%%%%%%%%%%%%%%%%%%%%%%%%%%%%%%%%%%%%%%%%%%%%%%%%%%%%%%%%%%
%%%%%        maths symbols                    	     %%%%%
%%%%%%%%%%%%%%%%%%%%%%%%%%%%%%%%%%%%%%%%%%%%%%%%%%%%%%%%%%%
\usepackage{amsopn}
\DeclareMathOperator{\ch}{ch}
\DeclareMathOperator{\sh}{sh}
\DeclareMathOperator{\argch}{argch}
\DeclareMathOperator{\argsh}{argsh}
\DeclareMathOperator{\argth}{argth}

\DeclareMathOperator{\sign}{sign}
\DeclareMathOperator{\Card}{Card}
\DeclareMathOperator{\rg}{rg}
\DeclareMathOperator{\tr}{tr}

\DeclareMathOperator{\corr}{corr}
\DeclareMathOperator{\var}{var}
\DeclareMathOperator{\vect}{vect}

\DeclareMathOperator{\cov}{cov}
\DeclareMathOperator{\pred}{pred}
\DeclareMathOperator{\Id}{Id}

\renewcommand{\Re}{\mathop{\mathrm{Re}}}
\newcommand{\argmin}{\mathop{\mathrm{arg\,min}}}
\newcommand{\argmax}{\mathop{\mathrm{arg\,max}}}
\newcommand{\spn}{\mathop{\mathrm{span}}}
\def\rang{\mathop{\rm rang}\nolimits}
\def\Ker{\mathop{\rm Ker}\nolimits}
\def\Im{\mathop{\rm Im}\nolimits}
\def\Vect{\mathop{\rm Vect}\nolimits}
\def\proj{\mathop{\rm proj}\nolimits}




\def\dim{\mathop{\rm dim}\nolimits}

\def\dom{\mathop{\rm dom}\nolimits}
\def\supp{\mathop{\rm supp}\nolimits}
\def\relint{\mathop{\rm relint}\nolimits}
\def\prox{\mathop{\rm prox}\nolimits}

\def\bfr{\mathbf{r}}
\def\bfx{\mathbf{x}}
\def\bfX{\mathbf{X}}

\def\bfy{\mathbf{y}}
\def\bftheta{\boldsymbol{\theta}}


\def\calA{\mathcal{A}}
\def\calB{\mathcal{B}}
\def\calC{\mathcal{C}}
\def\calD{\mathcal{D}}
\def\calE{\mathcal{E}}
\def\calF{\mathcal{F}}
\def\calG{\mathcal{G}}
\def\calH{\mathcal{H}}
\def\calI{\mathcal{I}}
\def\calJ{\mathcal{J}}
\def\calK{\mathcal{K}}
\def\calL{\mathcal{L}}
\def\calM{\mathcal{M}}
\def\calN{\mathcal{N}}
\def\calO{\mathcal{O}}
\def\calP{\mathcal{P}}
\def\calQ{\mathcal{Q}}
\def\calR{\mathcal{R}}
\def\calS{\mathcal{S}}
\def\calT{\mathcal{T}}
\def\calU{\mathcal{U}}
\def\calV{\mathcal{V}}
\def\calW{\mathcal{W}}
\def\calX{\mathcal{X}}
\def\calY{\mathcal{Y}}
\def\calZ{\mathcal{Z}}

\def\cH{\calH}
\def\cN{\calN}


\newcommand{\tnorm}[1]{{\left\vert\kern-0.25ex\left\vert\kern-0.25ex\left\vert #1 
    \right\vert\kern-0.25ex\right\vert\kern-0.25ex\right\vert}} % triple norm

\newcommand{\inner}[1]{{
\left\langle #1 \right\rangle}} % inner product

\newcommand{\norm}[1]{{
\left\| #1 \right\|}} % norm

\newcommand{\abs}[1]{{
\left| #1 \right|}} % abs

\newcommand{\piecewise}[1]{{
    \left\lbrace \begin{array}{ll} #1 \end{array} \right. }} % for piecewise functions

\newcommand{\fundef}[1]{{
\begin{array}{lcl} #1 \end{array} }} % for functions definitions

\newcommand{\seg}[1]{{\left[ #1 \right]}} % closed segment [ ]
\newcommand{\osego}[1]{{\left] #1 \right[}} % open segment ] [
\newcommand{\oseg}[1]{{\left] #1 \right]}} % semi-open segment ] ]
\newcommand{\sego}[1]{{\left[ #1 \right[}} % semi-open segment [ [
\newcommand{\iseg}[1]{{\left\llbracket #1 \right\rrbracket}} % integer segment

\newcommand{\ens}[1]{{
\left\lbrace #1 \right\rbrace }} % ensemble

\newcommand{\floor}[1]{{
\left\lfloor #1 \right\rfloor }} % floor

\newcommand{\pivot}[3]{{
    \left(
      \begin{array}{#1|#2}
        #3
      \end{array}
    \right)
  }}
\usepackage{color}
\newcommand{\red}{\color{red}}
\definecolor{gray}{rgb}{0.4, 0.4, 0.4}
\usepackage{sansmathaccent}
\pdfmapfile{+sansmathaccent.map}

\title[HMC]{Introduction aux méthodes de Monte Carlo par dynamique Hamiltonienne}
\author{Shmuel RAKOTONIRINA-RICQUEBOURG, Amaury DURAND}

\begin{document}
\begin{frame}
\titlepage
\end{frame}
\begin{frame}
  \frametitle{Plan}
  \tableofcontents[hideallsubsections]
\end{frame}

\section{Introduction}

\subsection{Algorithmes MCMC}

\begin{frame}
	\frametitle{Principe des MCMC}
	\begin{itemize}
		\item Objectif : pour $\pi$ à densité $h_\pi$ simuler $\pi$ ou approcher $\pi f$
		\item Idée : trouver une chaîne de Markov $X$ admettant $\pi$ comme loi invariante et convergeant vers $\pi$
	\end{itemize}
	\begin{Thm}[Théorème ergodique]\label{thm:ergodic}
		Soit $(X_k)_{k \in \nset}$ une chaîne de Markov de noyau $P$ sur $(\xset, \calX)$ admettant une unique loi invariante $\pi$. Alors pour tout $f \in \fset_+(\xset, \calX) \cup \fset_b(\xset, \calX)$ et pour $\pi$-presque tout $x \in \xset$,
		$$\frac{1}{n} \sum_{k=0}^{n-1} f(X_k) \xrightarrow[n \to +\infty]{\PP_x\text{-}\ps} \pi f$$
	\end{Thm}
\end{frame}

\subsection{Algorithme de Metropolis (Random Walk Metropolis)}

\begin{frame}
  \frametitle{Algorithme Random-walk Metropolis}
	\begin{center}
		\begin{algorithm}[H]
			\KwData{$h_\pi$ proportionnel à la densité cible, $Q$ loi simulable}
			$X_0 \leftarrow x \in \xset$ arbitraire\;
			$(U_k)_{k \in \nset} \siid Q$ \;
			\Repeat{une condition d'arrêt}{
				$Y_{k+1} \leftarrow X_k + U_{k+1}$ \tcp*{Proposer un mouvement}
				$\alpha_{k+1} \leftarrow \alpha(X_k, Y_{k+1})$ où $\alpha(x,y) = 1 \wedge \frac{h_{\pi}(y)}{h_\pi(x)}$\;
				$X_{k+1} \leftarrow \piecewise{ Y_{k+1} & \text{avec probabilité } \alpha_{k+1} \\ X_k & \text{with probability } 1 - \alpha_{k+1}}$ \tcp*{Accepter ou rejeter le mouvement}
			}
			\KwRet{$(X_k)_k$}
			\caption{Random Walk Metropolis}
			\label{algo:metropolis}
		\end{algorithm}
	\end{center}
\end{frame}

\section{Dynamique hamiltonienne}

\subsection{Définition}

\begin{frame}
	\frametitle{Dynamique hamiltonienne}
	\begin{Def}(Dynamique hamiltonienne)
		Soit $H : \fundef{\rset^d \times \rset^d & \to & \rset \\ (x,p) & \mapsto & H(x,p)}$. Deux fonctions de position $x : \rset_+ \to \rset^d$ et de quantité de mouvement $p : \rset_+ \to \rset^d$ sont dites solutions du hamiltonien $H$ (ou suivant la dynamique hamiltonienne de $H$) si
		\begin{equation*}\label{eq:hamiltonian-dyn}
			\begin{aligned}
				x'(t) &= \frac{\partial H}{\partial p} (x(t), p(t)) \\
				p' (t) &= -\frac{\partial H}{\partial x} (x(t), p(t))
			\end{aligned}
		\end{equation*}
	\end{Def}
	Pour $T>0$, $H$ peut être associée à la densité (sur $\rset^{2d}$)
	$$
	h(z) \propto \exp \left( -\frac{H(z)}{T} \right)
	$$
\end{frame}
 
\begin{frame}
	\frametitle{Hamiltonien pour l'algorithme HMC}
	Pour avoir $X \sim \pi$, on cherche à simuler $(X,P) \sim \widetilde \pi = \pi \otimes \nu$ pour une loi $\nu$ choisie. Dans toute la suite, on fera les hypothèses suivantes :
	\begin{Hyp}[H]\label{hyp:hyp}
		\begin{itemize}
			\item $\pi$ et $\nu$ sont à densité $h_\pi$ et $h_\nu$ strictement positives sur $\rset^d$
			\item $\exists k \geq 1, \ln(h_\pi)$ et $\ln(h_\nu)$ sont de classe $\calC^k$ sur $\rset^d$
			\item $h_\nu$ est paire
			\item $H : (x,p) \mapsto U(x) + K(p)$ avec $U = - T \ln(h_\pi), K = -T \ln(h_\nu)$
		\end{itemize}
	\end{Hyp}
	Ainsi, la densité $h \propto e^{H/T}$ est la densité jointe $h = h_\pi \otimes h_\nu = h_{\widetilde \pi}$.
\end{frame}
 
\subsection{Propriétés}

\begin{frame}
	\frametitle{Flot de l'équation différentielle}
	En notant $z = (x,p)$, les équations $x' = \frac{\partial H}{\partial p}, p' = -\frac{\partial H}{\partial x}$ se réécrivent $z' = F(z)$ où $F = J \nabla H$ et
	$J = \begin{bmatrix}
		0_{d} & I_{d} \\
		-I_{d} & 0_{d}
	\end{bmatrix}$.
	\begin{Prop}[conservation du hamiltonien]
		Le hamiltonien est conservé le long des trajectoires : si $z' = F(z)$, alors $H \circ z$ est constant.
	\end{Prop}
	\begin{Def}[flot hamiltonien]
		Pour $t \in \rset$, on définit le flot $\phi_t$ de sorte que pour tout $z_0 \in \rset^{2d}$, $t \mapsto \phi_t(z_0)$ soit l'unique solution du hamiltonien avec condition initiale $z(0) = z_0$.
	\end{Def}
	\begin{Prop}[conservation du volume]
		La solution du hamiltonien conserve le volume : $\detjacob{\phi_t}{z} = 1$.
	\end{Prop}
\end{frame}

\begin{frame}
	\frametitle{Réversibilité du flot}
	\begin{Lem}[réversibilité du temps]
		Soit $z = (x,p)$ une solution du hamiltonien $H$. On définit $\bar{z} = (\bar x, \bar p)$ par $\bar z(t) = (x(-t), -p(-t))$. Alors
		\begin{enumerate}
			\item $\bar{z}$ est solution du hamiltonien $H$ (avec d'autres conditions initiales).
			\item $\forall t \in \rset_+, \phi_t(\bar{z}(-t)) = \bar{z}(0)$
		\end{enumerate}
	\end{Lem}
	\begin{Prop}[réversibilité du flot]
		$\phi_t$ est un $\calC^k$-difféomorphisme d'inverse $\phi_t^{-1} : (x,p) \mapsto (\phi_t^{(1)}(x, -p), - \phi_t^{(2)}(x, -p))$.
	\end{Prop}
\end{frame}

\begin{frame}
	\begin{proof}[Preuve de la réversibilité du temps]
		On définit la symétrie $s(x,p) = (x,-p)$. Par définition, $\bar{z}(t) = s \circ z (-t)$ et par hypothèse de parité de $h_\nu$, $H = H \circ s$.

		Notons $S \doteq \frac{ds}{dz}(z) = \begin{bmatrix} I_d & 0_d \\ 0_d & -I_d \end{bmatrix}$ (et ce pour tout $z$). Ainsi, $\nabla H = S \nabla H \circ s$. On remarque que $SJ = -JS$ et $s^{-1} = s$.

		$$\bar z'(t) = -Sz'(-t) = -SJ\nabla H(z(-t)) = JS \nabla H(z(-t))$$
		donc
		$$\bar z'(t) = J \nabla H (s^{-1}(z(-t))) = J \nabla H(\bar z(t))$$
	\end{proof}
\end{frame}

\begin{frame}
	\begin{proof}[Preuve de la réversibilité du flot]
		Il suffit de prouver la formule de l'inverse. On pose $\bar \phi_t(x,p) = (\phi_t^{(1)}(x, -p), - \phi_t^{(2)}(x, -p))$. On fixe $z_0 = (x_0,p_0)$ et on note $z$ la solution du hamiltonien avec $z(0) = z_0$.
		\begin{itemize}
			\item D'une part,
			$$
			\bar \phi_t(\phi_t(x_0,p_0)) = \bar \phi_t(z(t)) = (\phi_t^{(1)}(\bar z(-t)), - \phi_t^{(2)}(\bar z(-t))) = (x_0, p_0)
			$$
			car $\bar z(0) = (x_0,-p_0)$.

			\item D'autre part,
			$$
			\bar \phi_t(x_0,-p_0) = (\phi_t^{(1)}(x_0, p_0), - \phi_t^{(2)}(x_0, p_0)) = (x(t),-p(t)) = \bar z(-t)
			$$
			donc
			$$
			\phi_t (\bar \phi_t(x_0,-p_0)) = \phi_t(\bar z(-t)) = \bar{z}(0) = (x_0,-p_0).
			$$
		\end{itemize}
	\end{proof}
\end{frame}
 
\subsection{Discrétisation}

\begin{frame}
	\frametitle{Algorithme du leapfrog}
	Variante de la méthode d'Euler : on discrétise $x' = \nabla K(p), p' = - \nabla U(x)$ en
	\begin{enumerate}
		\item $p_{t+\epsilon/2} = p_t - \frac{\epsilon}{2} \nabla U(x_t)$ (demi-pas en $p$).
		\item $x_{t+\epsilon} = p_t + \epsilon \nabla K(p_{t+\epsilon/2})$ (pas en $x$).
		\item $p_{t+\epsilon} = p_{t+\epsilon/2} - \frac{\epsilon}{2} \nabla U(x_{t+\epsilon})$ (demi-pas en $p$).
	\end{enumerate}
	\begin{center}
		\begin{algorithm}[H]
			\KwData{pas $\epsilon$, nombre de pas $L$, état initial $(x_0,p_0)$}
			\For(\tcp*[h]{Saute-mouton}){$k \in \iseg{0, L-1}$}{
				$x_{k+1} \leftarrow x_k + \epsilon \nabla K \left( p_k - \frac{\epsilon}{2} \nabla U(x_k) \right)$\;
				$p_{k+1} \leftarrow p_k - \frac{\epsilon}{2} \nabla U(x_k) - \frac{\epsilon}{2} \nabla U(x_{k+1})$\;
			}
			\KwRet{$(x_L,p_L)$}
			\caption{Discrétisation de l'évolution par saute-mouton ({\it leapfrog})}
			\label{algo:leapfrog}
		\end{algorithm}
	\end{center}
\end{frame}

\begin{frame}
	\frametitle{Flot approché}
	\begin{center} % je rappelle l'algorithme pour que le public voit d'où sort la déf
		\small
		\begin{algorithm}[H]
			\KwData{pas $\epsilon$, nombre de pas $L$, état initial $(x_0,p_0)$}
			\For(\tcp*[h]{Saute-mouton}){$k \in \iseg{0, L-1}$}{
				$x_{k+1} \leftarrow x_k + \epsilon \nabla K \left( p_k - \frac{\epsilon}{2} \nabla U(x_k) \right)$\;
				$p_{k+1} \leftarrow p_k - \frac{\epsilon}{2} \nabla U(x_k) - \frac{\epsilon}{2} \nabla U(x_{k+1})$\;
			}
			\KwRet{$(x_L,p_L)$}
			\label{algo:leapfrog}
		\end{algorithm}
	\end{center}
	\begin{Def}[flot approché] \label{def:flot_approche}
		On fixe $\epsilon > 0$. Pour $L \in \nset^*$, on définit le flot approché du hamiltonien par $L$ itérations de l'algorithme leapfrog par $\hat{\phi}_L = \hat{\phi}^L$ où $\hat \phi$ est défini par
		\begin{align*}
		\hat{\phi}^{(1)} : (x,p) & \mapsto x + \epsilon \nabla K \left( p - \frac{\epsilon}{2} \nabla U(x) \right) \\
		\hat{\phi}^{(2)} : (x,p) & \mapsto p - \frac{\epsilon}{2} \nabla U(x) - \frac{\epsilon}{2} \nabla U(\hat{\phi}^{(1)}(x,p))
		\end{align*}
	\end{Def}
\end{frame}

\begin{frame}
	\frametitle{Propriétés du flot approché}
	\begin{Prop}[conservation du volume]
		On suppose $k \geq 2$ ($U$ et $K$ de classe $\calC^2$). La solution approchée du hamiltonien par le leapfrog conserve le volume : $\detjacob{\hat \phi}{z} = 1$.
	\end{Prop}
	\begin{Prop}[réversibilité du flot approché]
		$\hat \phi$ est inversible d'inverse $\phi^{-1} = s \circ \hat \phi \circ s : (x,p) \mapsto (\hat{\phi}^{(1)}(x,-p), -\hat{\phi}^{(2)}(x,-p))$.
	\end{Prop}
\end{frame}
 
\section{Hamiltonian Monte-Carlo}

\subsection{Cas idéal}
 
% invariance de la loi
 
\begin{frame}
	\frametitle{Algorithme HMC idéal}
	\begin{center}
		\begin{algorithm}[H]
			\KwData{$h_\pi$ proportionnel à la densité cible, $t$ une durée sur laquelle suivre la dynamique}
			$X_0 \leftarrow x \in \xset$ arbitraire\;
			\Repeat{une condition d'arrêt}{
				$\tilde{P}_k \sim \mathcal \nu$ et $\tilde{P}_k \indep (Z_0, \cdots, Z_{k})$ \tcp*{Tirer la quantité de mouvement}
				$\tilde{Z}_k \leftarrow (X_k, \tilde{P}_k)$ \;
				$Z_{k+1} = (X_{k+1}, P_{k+1}) \leftarrow \phi_t(\tilde{Z}_k)$ \tcp*{Suivre la dynamique}
			}
			\KwRet{$(X_k)_k$}
			\caption{Hamiltonian Monte-Carlo, cas idéal}
			\label{algo:HMC-ideal}
		\end{algorithm}
	\end{center}
\end{frame}

\begin{frame}
	\frametitle{Invariance de $\widetilde \pi$}
	\begin{Prop}
		Pour $t \in \rset_+$, le processus $(Z_k)_{k \in \nset}$ défini par l'algorithme \ref{algo:HMC-ideal} est une chaîne de Markov homogène de noyau $QP_t$ sur $\rset^{2d}$ avec
		$$
		Q((x,p),\cdot) = \delta_x \otimes \nu \text{ et } P_t((x,p),\cdot) = \delta_{\phi_t(z)}.
		$$
		De plus, $\widetilde{\pi}$ est $Q$ et $P_t$-invariante. 
	\end{Prop}
	\begin{proof}[Preuve de la $P_t$-invariance]
		\small
		\begin{align*}
		\widetilde{\pi} P_t (A)
		&= \int \widetilde{\pi}(dz) P_t(z, A)
		= c \int \exp \left(-\frac{H(z)}{T} \right) \ind_A(\phi_t(z)) dz\\
		&= c \int \exp \left(-\frac{H \circ \phi_t^{-1}(y)}{T} \right) \ind_A(y) dy
		= c \int \exp \left(-\frac{H(y)}{T} \right) \ind_A(y) dy \\
		&= \widetilde{\pi}(A).
		\end{align*}
	\end{proof}
\end{frame}
 
\subsection{Cas réel}
 
% réversibilité de la loi

\begin{frame}
	\frametitle{Modification du HMC pour la discrétisation}
	Contrairement à $\phi_t$, $\hat \phi_L$ ne conserve pas le hamiltonien.

	Idée : remplacer $\phi_t$ par $g \circ \hat \phi_L$ pour $g$ une fonction à déterminer, i.e. remplacer $P_t$ par
	$$
	\hat{P}_L : \fundef{
		\rset^{2d} \times \calB(\rset^{2d}) & \to & [0,1] \\
		(z, A) & \mapsto & \delta_{g \circ \hat{\phi}_L(z)}(A)
	}
	$$

	% Problème : $\widetilde \pi$ est $Q$-invariante mais pas $\hat P_L$-invariante.

	\begin{Prop}[réversibilité]\label{prop:rever_discret}
		On suppose $k \geq 2$ ($U$ et $K$ de classe $\calC^2$). On prend $g = s$ et
		$$
		\alpha : (z_0,z_1) \mapsto 1 \wedge \frac{h_{\widetilde \pi}(z_1)}{h_{\widetilde \pi}(z_0)} = 1 \wedge \exp(\frac{-H(z_1)+H(z_0)}{T}).
		$$
		Si $\hat P_L^\alpha$ est le noyau de Metropolis-Hasting associé au noyau instrumental $\hat P_L$ et à la fonction de rejet $\alpha$, alors $\widetilde \pi$ est $\hat P_L^\alpha$-réversible (et donc $\hat P_L^\alpha$-invariant).
	\end{Prop}
\end{frame}

\begin{frame}
	\begin{proof}
		\small
		Notons $\psi_L = s \circ \hat \phi_L$, de sorte que $\hat P_L(z,\cdot) = \delta_{\psi_L(z)}$. Ainsi,
		$$
		\hat P_L^\alpha(z,A) = \alpha(z,\psi_L(z)) \ind_A(\psi_L(z)) + (1-\alpha(z,\psi_L(z))) \ind_A(z)
		$$
		donc
		$$
		\widetilde \pi \otimes \hat P_L^\alpha (A \times B)
		=
		\Lambda_1(A,B) + \Lambda_2(A,B)
		$$
		où
		$$
		\Lambda_2(A,B) = \int \ind_A(z) \ind_B(z) (1-\alpha(z,\psi_L(z))) e^{-H(z)/T} dz \text{ est sym\'etrique}
		$$
		et
		$$
		\begin{aligned}
		\Lambda_1(A,B)
		&= \int \ind_A(z) \ind_B(\psi_L(z)) \alpha(z,\psi_L(z)) e^{-H(z)/T} dz\\
		&= \int \ind_A(z) \ind_B(\psi_L(z)) \left( e^{-H(z)/T} \wedge e^{-H(\psi_L(z))/T} \right) dz.
		\end{aligned}
		$$

		$\psi_L^{-1} = \psi_L$ (réversibilité de $\hat \phi_L$) et $\detjacob{\psi_L}{z} = -1$ (conservation du volume) donc le changement de variable $z \mapsto \psi_L(z)$ donne que $\Lambda_1$ est symétrique.
	\end{proof}
\end{frame}
 
\begin{frame}
  \frametitle{Algorithme HMC réel}
        \begin{center}
        \small
		\begin{algorithm}[H]
			\KwData{$h_\pi$ proportionnel à la densité cible, $\epsilon$ pas du saute-mouton, $L$ nombre de pas du saute-mouton}
			$X_0 \leftarrow x \in \xset$ arbitraire\;
			\Repeat{une condition d'arrêt}{
				$P_k \sim \mathcal N(0,1)$\tcp*{Tirer la quantité de mouvement}
				$(X_{prop},P_{prop}) \leftarrow \texttt{leapfrog}(X_k,P_k)$\tcp*{Proposer un mouvement}
				$U_k \leftarrow U(X_k); K_k \leftarrow \norm{P_k}^2/2$\;
				$U_{prop} \leftarrow U(X_{prop}); K_{prop} \leftarrow \norm{P_{prop}}^2/2$\;
				\eIf{$\mathcal U([0,1]) < \exp(U_k-U_{prop}+K_k-K_{prop})$}{
					$X_{k+1} \leftarrow X_{prop}$ \tcp*{Accepter}
				}{
					$X_{k+1} \leftarrow X_k$ \tcp*{Rejeter}
				}
			}
			\KwRet{$(X_k)_k$}
			\caption{Hamiltonian Monte-Carlo}
			\label{algo:HMC}
		\end{algorithm}
        \end{center}
\end{frame}
 
\section{Simulations}
 
% résultats des simulations

\end{document}


