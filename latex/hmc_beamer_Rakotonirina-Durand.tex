\documentclass[10pt]{beamer}
%%%%%%%%%%%%%%%%%%%%%%%%%%%%%%%%%%%%%%%%%%%%%%%%%%%%%%%%%%%
%%%%% 	French, accents and font coding     	      %%%%%
%%%%%%%%%%%%%%%%%%%%%%%%%%%%%%%%%%%%%%%%%%%%%%%%%%%%%%%%%%%

\usepackage[french]{babel}
%\usepackage{enumitem} 
\usepackage{fancyhdr}
\usepackage[utf8]{inputenc}
\usepackage{bbm, amsmath, amssymb, amsthm, amsfonts}			% for theorem definitions


\usepackage{graphicx}			% for images and graphics
\usepackage{stmaryrd}			% for \llbracket symbols [[ ]]

%%%%%%%%%%%%%%%%%%%%%%%%%%%%%%%%%%%%%%%%%%%%%%%%%%%%%%%%%%%
%%%%%  		 Theorems etc.                        %%%%%
%%%%%%%%%%%%%%%%%%%%%%%%%%%%%%%%%%%%%%%%%%%%%%%%%%%%%%%%%%%

\newtheorem{Thm}{Théorème}[section]
\newtheorem{Cor}{Corolaire}[Thm]
\newtheorem{Prop}{Proposition}[section]
\newtheorem{Lem}{Lemme}[section]
\theoremstyle{definition}
\newtheorem{Def}{Définition}[section]
\newtheorem{Ex}{Exemple}[section]
\newtheorem{Exo}{Exercice}[section]
\theoremstyle{remark}
\newtheorem*{Rque}{Remarque}
\theoremstyle{definition}
\newtheorem{Algo}{Algorithme}[section]

%%%%%%%%%%%%%%%%%%%%%%%%%%%%%%%%%%%%%%%%%%%%%%%%%%%%%%%%%%%
%%%%%  		Hyperlinks                            %%%%%
%%%%%%%%%%%%%%%%%%%%%%%%%%%%%%%%%%%%%%%%%%%%%%%%%%%%%%%%%%%
\usepackage{hyperref}
\hypersetup{
    colorlinks=true,
    linkcolor=blue,
    filecolor=magenta,      
    urlcolor=cyan,
}
%%%%%%%%%%%%%%%%%%%%%%%%%%%%%%%%%%%%%%%%%%%%%%%%%%%%%%%%%
%%%%		For Algorithms
%%%%
%%%%%%%%%%%%%%%%%%%%%%%%%%%%%%%%%%%%%%%%%%%%%%%%%%%%%%%%%
\usepackage[ruled, vlined, french, onelanguage, algosection, nofillcomment, scleft]{algorithm2e}
\SetKwBlock{KwInit}{Initialization :}{endInit}


%%%%%%%%%%%%%%%%%%%%%%%%%%%%%%%%%%%%%%%%%%%%%%%%%%%%%%%%%%%
%%%%%  		Code			 %%%%%
%%%%%%%%%%%%%%%%%%%%%%%%%%%%%%%%%%%%%%%%%%%%%%%%%%%%%%%%%%%
\usepackage[section]{minted}
\usemintedstyle{borland}
\usepackage{color}
\definecolor{bg}{rgb}{0.95,0.95,0.95}

%%%%%%%%%%%%%%%%%%%%%%%%%%%%%%%%%%%%%%%%%%%%%%%%%%%%%%%%%
%%%%%		table of contents						%%%%%
%%%%%%%%%%%%%%%%%%%%%%%%%%%%%%%%%%%%%%%%%%%%%%%%%%%%%%%%%
\AtBeginSection{
	\begin{frame}
		\tableofcontents[currentsection, hideothersubsections]
	\end{frame}
}

%%%%%%%%%%%%%%%%%%%%%%%%%%%%%%%%%%%%%%%%%%%%%%%%%%%%%%%%%%%
%%%%%  		 latin abbreviations                  %%%%%
%%%%%%%%%%%%%%%%%%%%%%%%%%%%%%%%%%%%%%%%%%%%%%%%%%%%%%%%%%%
\newcommand{\ie}{{\em i.e.,~}}
\newcommand{\eg}{{\em e.g.,~}}
\newcommand{\lcf}{{\em cf.~}}



%%%%%%%%%%%%%%%%%%%%%%%%%%%%%%%%%%%%%%%%%%%%%%%%%%%%%%%%%%%
%%%%%  		 Style math                         %%%%%
%%%%%%%%%%%%%%%%%%%%%%%%%%%%%%%%%%%%%%%%%%%%%%%%%%%%%%%%%%%

\newcommand{\nset}{\mathbb{N}}
\newcommand{\zset}{\mathbb{Z}}
\newcommand{\rset}{\mathbb{R}}
\newcommand{\cset}{\mathbb{C}}
\newcommand{\qset}{\mathbb{Q}}
\newcommand{\kset}{\mathbb{K}}
\newcommand{\tset}{\mathbb{T}}
\newcommand{\xset}{\mathbb{X}}
\newcommand{\yset}{\mathbb{Y}}
\newcommand{\uset}{\mathbb{U}}
\newcommand{\fset}{\mathbb{F}}
\newcommand{\mset}{\mathbb{M}}

\newcommand{\ps}{\text{ p.s. }}
\newcommand{\geqrespleq}{\underset{\mbox{ (resp $\leq$) }}{\geq}}
\newcommand{\leqrespgeq}{\underset{\mbox{ (resp $\geq$) }}{\leq}}



\usepackage{xifthen}
%\newcommand{\EE}[1][]{%
%  \ifthenelse{\isempty{#1}}%
%    {\mathbb{E}}% if #1 is empty
%    {\mathbb{E}\left[#1\right]}% if #1 is not empty
%}

%\newcommand{\PP}[1][]{%
%  \ifthenelse{\isempty{#1}}%
%    {\mathbb{P}}% if #1 is empty
%    {\mathbb{P}\left(#1\right)}% if #1 is not empty
%}

\newcommand{\law}[1][]{%
  \ifthenelse{\isempty{#1}}%
    {\calL}% if #1 is empty
    {\calL \left(#1\right)}% if #1 is not empty
}

%\newcommand{\pp}[1][]{%
%  \ifthenelse{\isempty{#1}}%
%    {p}% if #1 is empty
%    {p\left(#1\right)}% if #1 is not empty
%}
% 
%\newcommand{\Cov}[1][]{%
%  \ifthenelse{\isempty{#1}}%
%    {\mathrm{Cov}}% if #1 is empty
%    {\mathrm{Cov}\left(#1\right)}% if #1 is not empty
%}
% 
%\newcommand{\Var}[1][]{%
%  \ifthenelse{\isempty{#1}}%
%    {\mathrm{Var}}% if #1 is empty
%    {\mathrm{Var}\left(#1\right)}% if #1 is not empty
%  }

\usepackage{xparse}
\DeclareDocumentCommand{\PP}{go}
{\IfNoValueTF{#2}
  {\IfNoValueTF{#1}
    {\mathbb{P}}
    {\mathbb{P}_{#1}}%
  }
  {\IfNoValueTF{#1}
    {\mathbb{P}\left(#2\right)}
    {\mathbb{P}_{#1}\left(#2\right)}%
  }%
}

\DeclareDocumentCommand{\EE}{go}
{\IfNoValueTF{#2}
  {\IfNoValueTF{#1}
    {\mathbb{E}}
    {\mathbb{E}_{#1}}%
  }
  {\IfNoValueTF{#1}
    {\mathbb{E}\left[#2\right]}
    {\mathbb{E}_{#1}\left[#2\right]}%
  }%
}

\DeclareDocumentCommand{\pp}{go}
{\IfNoValueTF{#2}
  {\IfNoValueTF{#1}
    {p}
    {p_{#1}}%
  }
  {\IfNoValueTF{#1}
    {p\left(#2\right)}
    {p_{#1}\left(#2\right)}%
  }%
}

\DeclareDocumentCommand{\Cov}{go}
{\IfNoValueTF{#2}
  {\IfNoValueTF{#1}
    {\mathrm{Cov}}
    {\mathrm{Cov}_{#1}}%
  }
  {\IfNoValueTF{#1}
    {\mathrm{Cov}\left(#2\right)}
    {\mathrm{Cov}_{#1}\left(#2\right)}%
  }%
}

\DeclareDocumentCommand{\Var}{go}
{\IfNoValueTF{#2}
  {\IfNoValueTF{#1}
    {\mathrm{Var}}
    {\mathrm{Var}_{#1}}%
  }
  {\IfNoValueTF{#1}
    {\mathrm{Var}\left(#2\right)}
    {\mathrm{Var}_{#1}\left(#2\right)}%
  }%
}



\def\1{\mathbf 1}
\def\ind{\mathbbm 1}
\newcommand{\mb}{\mathbf}
\newcommand{\ud}{\,\mathrm{d}} 
\newcommand{\given}[1][{}]{\;\middle\vert\;{#1} }
\def\indep{\perp\!\!\!\perp}


%%%%%%%%%%%%%%%%%%%%%%%%%%%%%%%%%%%%%%%%%%%%%%%%%%%%%%%%%%%
%%%%%        equality symbols                        %%%%%
%%%%%%%%%%%%%%%%%%%%%%%%%%%%%%%%%%%%%%%%%%%%%%%%%%%%%%%%%%%
\newcommand{\deq}{\stackrel{\rm def}{=}}
\newcommand{\eqlaw}{\stackrel{\rm loi}{=}}
\newcommand{\siid}{\stackrel{\rm iid}{\sim}}
\newcommand{\cps}{\xrightarrow[n \to +\infty]{\ps}}
\newcommand{\cprob}{\xrightarrow[n \to +\infty]{P}}
\newcommand{\claw}{\xrightarrow[n \to +\infty]{\calL}}


%%%%%%%%%%%%%%%%%%%%%%%%%%%%%%%%%%%%%%%%%%%%%%%%%%%%%%%%%%%
%%%%%        maths symbols                    	     %%%%%
%%%%%%%%%%%%%%%%%%%%%%%%%%%%%%%%%%%%%%%%%%%%%%%%%%%%%%%%%%%
\usepackage{amsopn}
\DeclareMathOperator{\ch}{ch}
\DeclareMathOperator{\sh}{sh}
\DeclareMathOperator{\argch}{argch}
\DeclareMathOperator{\argsh}{argsh}
\DeclareMathOperator{\argth}{argth}

\DeclareMathOperator{\sign}{sign}
\DeclareMathOperator{\Card}{Card}
\DeclareMathOperator{\rg}{rg}
\DeclareMathOperator{\tr}{tr}

\DeclareMathOperator{\corr}{corr}
\DeclareMathOperator{\var}{var}
\DeclareMathOperator{\vect}{vect}

\DeclareMathOperator{\cov}{cov}
\DeclareMathOperator{\pred}{pred}
\DeclareMathOperator{\Id}{Id}

\renewcommand{\Re}{\mathop{\mathrm{Re}}}
\newcommand{\argmin}{\mathop{\mathrm{arg\,min}}}
\newcommand{\argmax}{\mathop{\mathrm{arg\,max}}}
\newcommand{\spn}{\mathop{\mathrm{span}}}
\def\rang{\mathop{\rm rang}\nolimits}
\def\Ker{\mathop{\rm Ker}\nolimits}
\def\Im{\mathop{\rm Im}\nolimits}
\def\Vect{\mathop{\rm Vect}\nolimits}
\def\proj{\mathop{\rm proj}\nolimits}




\def\dim{\mathop{\rm dim}\nolimits}

\def\dom{\mathop{\rm dom}\nolimits}
\def\supp{\mathop{\rm supp}\nolimits}
\def\relint{\mathop{\rm relint}\nolimits}
\def\prox{\mathop{\rm prox}\nolimits}

\def\bfr{\mathbf{r}}
\def\bfx{\mathbf{x}}
\def\bfX{\mathbf{X}}

\def\bfy{\mathbf{y}}
\def\bftheta{\boldsymbol{\theta}}


\def\calA{\mathcal{A}}
\def\calB{\mathcal{B}}
\def\calC{\mathcal{C}}
\def\calD{\mathcal{D}}
\def\calE{\mathcal{E}}
\def\calF{\mathcal{F}}
\def\calG{\mathcal{G}}
\def\calH{\mathcal{H}}
\def\calI{\mathcal{I}}
\def\calJ{\mathcal{J}}
\def\calK{\mathcal{K}}
\def\calL{\mathcal{L}}
\def\calM{\mathcal{M}}
\def\calN{\mathcal{N}}
\def\calO{\mathcal{O}}
\def\calP{\mathcal{P}}
\def\calQ{\mathcal{Q}}
\def\calR{\mathcal{R}}
\def\calS{\mathcal{S}}
\def\calT{\mathcal{T}}
\def\calU{\mathcal{U}}
\def\calV{\mathcal{V}}
\def\calW{\mathcal{W}}
\def\calX{\mathcal{X}}
\def\calY{\mathcal{Y}}
\def\calZ{\mathcal{Z}}

\def\cH{\calH}
\def\cN{\calN}


\newcommand{\tnorm}[1]{{\left\vert\kern-0.25ex\left\vert\kern-0.25ex\left\vert #1 
    \right\vert\kern-0.25ex\right\vert\kern-0.25ex\right\vert}} % triple norm

\newcommand{\inner}[1]{{
\left\langle #1 \right\rangle}} % inner product

\newcommand{\norm}[1]{{
\left\| #1 \right\|}} % norm

\newcommand{\abs}[1]{{
\left| #1 \right|}} % abs

\newcommand{\piecewise}[1]{{
    \left\lbrace \begin{array}{ll} #1 \end{array} \right. }} % for piecewise functions

\newcommand{\fundef}[1]{{
\begin{array}{lcl} #1 \end{array} }} % for functions definitions

\newcommand{\seg}[1]{{\left[ #1 \right]}} % closed segment [ ]
\newcommand{\osego}[1]{{\left] #1 \right[}} % open segment ] [
\newcommand{\oseg}[1]{{\left] #1 \right]}} % semi-open segment ] ]
\newcommand{\sego}[1]{{\left[ #1 \right[}} % semi-open segment [ [
\newcommand{\iseg}[1]{{\left\llbracket #1 \right\rrbracket}} % integer segment

\newcommand{\ens}[1]{{
\left\lbrace #1 \right\rbrace }} % ensemble

\newcommand{\floor}[1]{{
\left\lfloor #1 \right\rfloor }} % floor

\newcommand{\pivot}[3]{{
    \left(
      \begin{array}{#1|#2}
        #3
      \end{array}
    \right)
  }}
\usepackage{color}
\newcommand{\red}{\color{red}}
\definecolor{gray}{rgb}{0.4, 0.4, 0.4}

\title[HMC]{Introduction aux méthodes de Monte Carlo par dynamique Hamiltonienne}
\author{Shmuel RAKOTONIRINA-RICQUEBOURG, Amaury DURAND}

\begin{document}
\begin{frame}
\titlepage
\end{frame}
\begin{frame}
  \frametitle{Plan}
  \tableofcontents[hideallsubsections]
\end{frame}

\section{Introduction}

\subsubsection{Algorithmes MCMC}

% contextualisation : principe des MCMC

\subsubsection{Algorithme de Metropolis (Random Walk Metropolis)}

\begin{frame}
  \frametitle{Algorithme Random-walk Metropolis}
	\begin{center}
		\begin{algorithm}[H]
			\KwData{$h_\pi$ proportionnel à la densité cible, $Q$ loi simulable}
			$X_0 \leftarrow x \in \xset$ arbitraire\;
			$(U_k)_{k \in \nset} \siid Q$ \;
			\Repeat{une condition d'arrêt}{
				$Y_{k+1} \leftarrow X_k + U_{k+1}$ \tcp*{Proposer un mouvement}
				$\alpha_{k+1} \leftarrow \alpha(X_k, Y_{k+1})$ où $\alpha(x,y) = 1 \wedge \frac{h_{\pi}(y)}{h_\pi(x)}$\;
				$X_{k+1} \leftarrow \piecewise{ Y_{k+1} & \text{avec probabilité } \alpha_{k+1} \\ X_k & \text{with probability } 1 - \alpha_{k+1}}$ \tcp*{Accepter ou rejeter le mouvement}
			}
			\KwRet{$(X_k)_k$}
			\caption{Random Walk Metropolis}
			\label{algo:metropolis}
		\end{algorithm}
	\end{center}
\end{frame}

\section{Dynamique hamiltonienne}
\subsection{Définition}
 
% définition rapide dynamique hamiltonienne
 
\begin{frame}
  \frametitle{Hypothèses}
        \begin{equation}\label{eq:hyp}
        	\begin{aligned}
        		&\mbox{$h_\pi$ et $h_\nu$ sont strictement positives sur $\rset^d$} \\
        		&\mbox{$\exists k \geq 1, \ln(h_\pi)$ et $\ln(h_\nu)$ sont de classe $\calC^k$ sur $\rset^d$}\\
        		&\mbox{$h_\nu$ est paire} \\
                        & H : (x,p) \mapsto U(x) + K(p) \text{ avec } U(x) = - T \ln(h_\pi(x)), K(p) = -T \ln(h_\nu(p))
        	\end{aligned}
        	\tag{H}
        \end{equation}
\end{frame}
 
\subsection{Propriétés}

% conservation du hamiltonien, définition du flot, propriété de réversibilité du flot, conservation du volume
 
\subsection{Discrétisation}
 
% parler d'Euler à l'oral, définition du flot approché, réversibilité du flot, conservation du volume, 
\begin{frame}
  \frametitle{Algorithme du leapfrog}
        \begin{center}
                \begin{algorithm}[H]
                	\KwData{pas $\epsilon$, nombre de pas $L$, état initial $(x_0,p_0)$}
                	\For(\tcp*[h]{Saute-mouton}){$k \in \iseg{0, L-1}$}{
                		$x_{k+1} \leftarrow x_k + \epsilon \nabla K \left( p_k - \frac{\epsilon}{2} \nabla U(x_k) \right)$\;
                		$p_{k+1} \leftarrow p_k - \frac{\epsilon}{2} \nabla U(x_k) - \frac{\epsilon}{2} \nabla U(x_{k+1})$\;
                	}
                	\KwRet{$(x_L,p_L)$}
                	\caption{Discrétisation de l'évolution par saute-mouton ({\it leapfrog})}
                	\label{algo:leapfrog}
                \end{algorithm}
        \end{center}
\end{frame}
 
\section{Hamiltonian Monte-Carlo}
 
\subsection{Cas idéal}
 
% utilisation du flot exact, invariance de la loi
 
\begin{frame}
  \frametitle{Algorithme HMC idéal}
        \begin{center}
        	\begin{algorithm}[H]
        		\KwData{$h_\pi$ proportionnel à la densité cible, $t$ une durée sur laquelle suivre la dynamique}
        		$X_0 \leftarrow x \in \xset$ arbitraire\;
        		\Repeat{une condition d'arrêt}{
        			$\tilde{P}_k \sim \mathcal \nu$ et $\tilde{P}_k \indep (Z_0, \cdots, Z_{k})$ \tcp*{Tirer la quantité de mouvement}
        			$\tilde{Z}_k \leftarrow (X_k, \tilde{P}_k)$ \;
        			$Z_{k+1} = (X_{k+1}, P_{k+1}) \leftarrow \phi_t(\tilde{Z}_k)$ \tcp*{Suivre la dynamique}
        		}
        		\KwRet{$(X_k)_k$}
        		\caption{Hamiltonian Monte-Carlo, cas idéal}
        		\label{algo:HMC-ideal}
        	\end{algorithm}
        \end{center}
\end{frame}
 
\subsection{Cas réel}
 
% utilisation du flot approché, réversibilité de la loi
 
\begin{frame}
  \frametitle{Algorithme HMC réel}
        \begin{center}
        	\begin{algorithm}[H]
        		\KwData{$h_\pi$ proportionnel à la densité cible, $\epsilon$ pas du saute-mouton, $L$ nombre de pas du saute-mouton}
        		$X_0 \leftarrow x \in \xset$ arbitraire\;
        		\Repeat{une condition d'arrêt}{
        			$P_k \sim \mathcal N(0,1)$\tcp*{Tirer la quantité de mouvement}
        			$(X_{prop},P_{prop}) \leftarrow \texttt{leapfrog}(X_k,P_k)$\tcp*{Proposer un mouvement}
        			$U_k \leftarrow U(X_k)$\;
        			$K_k \leftarrow \norm{P_k}^2/2$\;
        			$U_{prop} \leftarrow U(X_{prop})$\;
        			$K_{prop} \leftarrow \norm{P_{prop}}^2/2$\;
        			\eIf{$\mathcal U([0,1]) < \exp(U_k-U_{prop}+K_k-K_{prop})$}{
        				$X_{k+1} \leftarrow X_{prop}$ \tcp*{Accepter}
        			}{
        				$X_{k+1} \leftarrow X_k$ \tcp*{Rejeter}
        			}
        		}
        		\KwRet{$(X_k)_k$}
        		\caption{Hamiltonian Monte-Carlo}
        		\label{algo:HMC}
        	\end{algorithm}
        \end{center}
\end{frame}
 
\section{Simulations}
 
% résultats des simulations

\end{document}


